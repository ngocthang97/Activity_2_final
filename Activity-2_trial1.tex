% Options for packages loaded elsewhere
\PassOptionsToPackage{unicode}{hyperref}
\PassOptionsToPackage{hyphens}{url}
\documentclass[
]{article}
\usepackage{xcolor}
\usepackage[margin=1in]{geometry}
\usepackage{amsmath,amssymb}
\setcounter{secnumdepth}{-\maxdimen} % remove section numbering
\usepackage{iftex}
\ifPDFTeX
  \usepackage[T1]{fontenc}
  \usepackage[utf8]{inputenc}
  \usepackage{textcomp} % provide euro and other symbols
\else % if luatex or xetex
  \usepackage{unicode-math} % this also loads fontspec
  \defaultfontfeatures{Scale=MatchLowercase}
  \defaultfontfeatures[\rmfamily]{Ligatures=TeX,Scale=1}
\fi
\usepackage{lmodern}
\ifPDFTeX\else
  % xetex/luatex font selection
\fi
% Use upquote if available, for straight quotes in verbatim environments
\IfFileExists{upquote.sty}{\usepackage{upquote}}{}
\IfFileExists{microtype.sty}{% use microtype if available
  \usepackage[]{microtype}
  \UseMicrotypeSet[protrusion]{basicmath} % disable protrusion for tt fonts
}{}
\makeatletter
\@ifundefined{KOMAClassName}{% if non-KOMA class
  \IfFileExists{parskip.sty}{%
    \usepackage{parskip}
  }{% else
    \setlength{\parindent}{0pt}
    \setlength{\parskip}{6pt plus 2pt minus 1pt}}
}{% if KOMA class
  \KOMAoptions{parskip=half}}
\makeatother
\usepackage{color}
\usepackage{fancyvrb}
\newcommand{\VerbBar}{|}
\newcommand{\VERB}{\Verb[commandchars=\\\{\}]}
\DefineVerbatimEnvironment{Highlighting}{Verbatim}{commandchars=\\\{\}}
% Add ',fontsize=\small' for more characters per line
\usepackage{framed}
\definecolor{shadecolor}{RGB}{248,248,248}
\newenvironment{Shaded}{\begin{snugshade}}{\end{snugshade}}
\newcommand{\AlertTok}[1]{\textcolor[rgb]{0.94,0.16,0.16}{#1}}
\newcommand{\AnnotationTok}[1]{\textcolor[rgb]{0.56,0.35,0.01}{\textbf{\textit{#1}}}}
\newcommand{\AttributeTok}[1]{\textcolor[rgb]{0.13,0.29,0.53}{#1}}
\newcommand{\BaseNTok}[1]{\textcolor[rgb]{0.00,0.00,0.81}{#1}}
\newcommand{\BuiltInTok}[1]{#1}
\newcommand{\CharTok}[1]{\textcolor[rgb]{0.31,0.60,0.02}{#1}}
\newcommand{\CommentTok}[1]{\textcolor[rgb]{0.56,0.35,0.01}{\textit{#1}}}
\newcommand{\CommentVarTok}[1]{\textcolor[rgb]{0.56,0.35,0.01}{\textbf{\textit{#1}}}}
\newcommand{\ConstantTok}[1]{\textcolor[rgb]{0.56,0.35,0.01}{#1}}
\newcommand{\ControlFlowTok}[1]{\textcolor[rgb]{0.13,0.29,0.53}{\textbf{#1}}}
\newcommand{\DataTypeTok}[1]{\textcolor[rgb]{0.13,0.29,0.53}{#1}}
\newcommand{\DecValTok}[1]{\textcolor[rgb]{0.00,0.00,0.81}{#1}}
\newcommand{\DocumentationTok}[1]{\textcolor[rgb]{0.56,0.35,0.01}{\textbf{\textit{#1}}}}
\newcommand{\ErrorTok}[1]{\textcolor[rgb]{0.64,0.00,0.00}{\textbf{#1}}}
\newcommand{\ExtensionTok}[1]{#1}
\newcommand{\FloatTok}[1]{\textcolor[rgb]{0.00,0.00,0.81}{#1}}
\newcommand{\FunctionTok}[1]{\textcolor[rgb]{0.13,0.29,0.53}{\textbf{#1}}}
\newcommand{\ImportTok}[1]{#1}
\newcommand{\InformationTok}[1]{\textcolor[rgb]{0.56,0.35,0.01}{\textbf{\textit{#1}}}}
\newcommand{\KeywordTok}[1]{\textcolor[rgb]{0.13,0.29,0.53}{\textbf{#1}}}
\newcommand{\NormalTok}[1]{#1}
\newcommand{\OperatorTok}[1]{\textcolor[rgb]{0.81,0.36,0.00}{\textbf{#1}}}
\newcommand{\OtherTok}[1]{\textcolor[rgb]{0.56,0.35,0.01}{#1}}
\newcommand{\PreprocessorTok}[1]{\textcolor[rgb]{0.56,0.35,0.01}{\textit{#1}}}
\newcommand{\RegionMarkerTok}[1]{#1}
\newcommand{\SpecialCharTok}[1]{\textcolor[rgb]{0.81,0.36,0.00}{\textbf{#1}}}
\newcommand{\SpecialStringTok}[1]{\textcolor[rgb]{0.31,0.60,0.02}{#1}}
\newcommand{\StringTok}[1]{\textcolor[rgb]{0.31,0.60,0.02}{#1}}
\newcommand{\VariableTok}[1]{\textcolor[rgb]{0.00,0.00,0.00}{#1}}
\newcommand{\VerbatimStringTok}[1]{\textcolor[rgb]{0.31,0.60,0.02}{#1}}
\newcommand{\WarningTok}[1]{\textcolor[rgb]{0.56,0.35,0.01}{\textbf{\textit{#1}}}}
\usepackage{graphicx}
\makeatletter
\newsavebox\pandoc@box
\newcommand*\pandocbounded[1]{% scales image to fit in text height/width
  \sbox\pandoc@box{#1}%
  \Gscale@div\@tempa{\textheight}{\dimexpr\ht\pandoc@box+\dp\pandoc@box\relax}%
  \Gscale@div\@tempb{\linewidth}{\wd\pandoc@box}%
  \ifdim\@tempb\p@<\@tempa\p@\let\@tempa\@tempb\fi% select the smaller of both
  \ifdim\@tempa\p@<\p@\scalebox{\@tempa}{\usebox\pandoc@box}%
  \else\usebox{\pandoc@box}%
  \fi%
}
% Set default figure placement to htbp
\def\fps@figure{htbp}
\makeatother
\setlength{\emergencystretch}{3em} % prevent overfull lines
\providecommand{\tightlist}{%
  \setlength{\itemsep}{0pt}\setlength{\parskip}{0pt}}
\usepackage{float}
\usepackage{booktabs}
\usepackage{caption}
\usepackage{longtable}
\usepackage{colortbl}
\usepackage{array}
\usepackage{anyfontsize}
\usepackage{multirow}
\usepackage{bookmark}
\IfFileExists{xurl.sty}{\usepackage{xurl}}{} % add URL line breaks if available
\urlstyle{same}
\hypersetup{
  pdftitle={Thang\_Activity 2},
  hidelinks,
  pdfcreator={LaTeX via pandoc}}

\title{Thang\_Activity 2}
\author{}
\date{\vspace{-2.5em}2025-10-24}

\begin{document}
\maketitle

\begin{Shaded}
\begin{Highlighting}[]
\DocumentationTok{\#\# Set global CRAN mirror}
\FunctionTok{options}\NormalTok{(}\AttributeTok{repos =} \FunctionTok{c}\NormalTok{(}\AttributeTok{CRAN =} \StringTok{"https://cran.rstudio.com/"}\NormalTok{))}

\DocumentationTok{\#\# Install the tableone package}
\FunctionTok{install.packages}\NormalTok{(}\StringTok{"tableone"}\NormalTok{)}
\end{Highlighting}
\end{Shaded}

\begin{verbatim}
## Installing package into 'C:/Users/Thang/AppData/Local/R/win-library/4.3'
## (as 'lib' is unspecified)
\end{verbatim}

\begin{verbatim}
## package 'tableone' successfully unpacked and MD5 sums checked
## 
## The downloaded binary packages are in
##  C:\Users\Thang\AppData\Local\Temp\RtmpyY9ShT\downloaded_packages
\end{verbatim}

\begin{Shaded}
\begin{Highlighting}[]
\DocumentationTok{\#\# Load the package}
\FunctionTok{library}\NormalTok{(tableone)}
\FunctionTok{library}\NormalTok{(readxl)}
\FunctionTok{library}\NormalTok{(dplyr)}
\end{Highlighting}
\end{Shaded}

\begin{verbatim}
## 
## Attaching package: 'dplyr'
\end{verbatim}

\begin{verbatim}
## The following objects are masked from 'package:stats':
## 
##     filter, lag
\end{verbatim}

\begin{verbatim}
## The following objects are masked from 'package:base':
## 
##     intersect, setdiff, setequal, union
\end{verbatim}

\begin{Shaded}
\begin{Highlighting}[]
\FunctionTok{library}\NormalTok{(janitor)}
\end{Highlighting}
\end{Shaded}

\begin{verbatim}
## 
## Attaching package: 'janitor'
\end{verbatim}

\begin{verbatim}
## The following objects are masked from 'package:stats':
## 
##     chisq.test, fisher.test
\end{verbatim}

\begin{Shaded}
\begin{Highlighting}[]
\FunctionTok{setwd}\NormalTok{(}\StringTok{"C:/Users/Thang/OneDrive/Desktop/Activity2"}\NormalTok{)}
\NormalTok{Data }\OtherTok{\textless{}{-}} \FunctionTok{read\_excel}\NormalTok{(}\StringTok{"CPAPAdherence\_Data\_Clean.xlsx"}\NormalTok{)}
\FunctionTok{str}\NormalTok{(Data)}
\end{Highlighting}
\end{Shaded}

\begin{verbatim}
## tibble [174 x 15] (S3: tbl_df/tbl/data.frame)
##  $ subject_id    : chr [1:174] "11-01102" "11-01153" "11-01442" "11-01634" ...
##  $ ethnicity     : chr [1:174] "Not Hispanic or Latino" "Not Hispanic or Latino" "Not Hispanic or Latino" "Not Hispanic or Latino" ...
##  $ education     : chr [1:174] "> high school" "> high school" "> high school" "> high school" ...
##  $ race          : chr [1:174] "Black" "White" "White" "White" ...
##  $ age           : num [1:174] 62 71 75 62 55 70 67 75 75 63 ...
##  $ sex           : chr [1:174] "Female" "Male" "Female" "Male" ...
##  $ bmi           : num [1:174] 51 42.8 53.5 35.8 41.4 ...
##  $ ahi           : num [1:174] 22.4 24.4 19.9 21.5 18.2 33.4 15.3 22 79.7 37 ...
##  $ ess           : num [1:174] 17 6 4 9 7 8 1 5 6 22 ...
##  $ mmse          : num [1:174] 27 30 30 29 29 26 29 29 30 26 ...
##  $ avg_daily_cpap: num [1:174] 6.45 9.05 4.57 7.62 6.3 ...
##  $ adherence     : chr [1:174] "Adherent" "Adherent" "Adherent" "Adherent" ...
##  $ odsi_bl       : num [1:174] 5 0 2 16 10 2 0 0 4 19 ...
##  $ odsi_6m       : num [1:174] 4 0 2 0 8 2 0 1 3 18 ...
##  $ adcs_12m      : num [1:174] 4 2 4 1 1 2 4 1 1 6 ...
\end{verbatim}

\begin{Shaded}
\begin{Highlighting}[]
\FunctionTok{head}\NormalTok{(Data)}
\end{Highlighting}
\end{Shaded}

\begin{verbatim}
## # A tibble: 6 x 15
##   subject_id ethnicity       education race    age sex     bmi   ahi   ess  mmse
##   <chr>      <chr>           <chr>     <chr> <dbl> <chr> <dbl> <dbl> <dbl> <dbl>
## 1 11-01102   Not Hispanic o~ > high s~ Black    62 Fema~  51.0  22.4    17    27
## 2 11-01153   Not Hispanic o~ > high s~ White    71 Male   42.8  24.4     6    30
## 3 11-01442   Not Hispanic o~ > high s~ White    75 Fema~  53.5  19.9     4    30
## 4 11-01634   Not Hispanic o~ > high s~ White    62 Male   35.8  21.5     9    29
## 5 11-01769   Not Hispanic o~ > high s~ Black    55 Fema~  41.4  18.2     7    29
## 6 11-01777   Not Hispanic o~ > high s~ White    70 Fema~  37.5  33.4     8    26
## # i 5 more variables: avg_daily_cpap <dbl>, adherence <chr>, odsi_bl <dbl>,
## #   odsi_6m <dbl>, adcs_12m <dbl>
\end{verbatim}

\begin{Shaded}
\begin{Highlighting}[]
\FunctionTok{dim}\NormalTok{(Data)}
\end{Highlighting}
\end{Shaded}

\begin{verbatim}
## [1] 174  15
\end{verbatim}

\begin{Shaded}
\begin{Highlighting}[]
\NormalTok{Data }\OtherTok{\textless{}{-}} \FunctionTok{clean\_names}\NormalTok{(Data)}

\NormalTok{Data }\OtherTok{\textless{}{-}}\NormalTok{ Data }\SpecialCharTok{\%\textgreater{}\%}
  \FunctionTok{mutate}\NormalTok{(}
    \AttributeTok{ethnicity =} \FunctionTok{factor}\NormalTok{(ethnicity),}
    \AttributeTok{education =} \FunctionTok{factor}\NormalTok{(education, }\AttributeTok{levels =} \FunctionTok{c}\NormalTok{(}\StringTok{"\textless{}= high school"}\NormalTok{, }\StringTok{"\textgreater{} high school"}\NormalTok{)),}
    \AttributeTok{race =} \FunctionTok{factor}\NormalTok{(race, }\AttributeTok{levels =} \FunctionTok{c}\NormalTok{(}\StringTok{"White"}\NormalTok{, }\StringTok{"Black"}\NormalTok{, }\StringTok{"Other"}\NormalTok{)),}
    \AttributeTok{adherence =} \FunctionTok{factor}\NormalTok{(adherence, }\AttributeTok{levels =} \FunctionTok{c}\NormalTok{(}\StringTok{"Non{-}adherent"}\NormalTok{, }\StringTok{"Adherent"}\NormalTok{))}
\NormalTok{  )}

\NormalTok{Data }\OtherTok{\textless{}{-}}\NormalTok{ Data }\SpecialCharTok{\%\textgreater{}\%}
  \FunctionTok{mutate}\NormalTok{(}
    \AttributeTok{age =} \FunctionTok{as.numeric}\NormalTok{(age),}
    \AttributeTok{ahi =} \FunctionTok{as.numeric}\NormalTok{(ahi),}
    \AttributeTok{ess =} \FunctionTok{as.numeric}\NormalTok{(ess),}
    \AttributeTok{mmse =} \FunctionTok{as.numeric}\NormalTok{(mmse),}
    \AttributeTok{avg\_daily\_cpap =} \FunctionTok{as.numeric}\NormalTok{(avg\_daily\_cpap)}
\NormalTok{  )}
\DocumentationTok{\#\# {-}{-}{-} Check for missing values {-}{-}{-}}
\FunctionTok{colSums}\NormalTok{(}\FunctionTok{is.na}\NormalTok{(Data))}
\end{Highlighting}
\end{Shaded}

\begin{verbatim}
##     subject_id      ethnicity      education           race            age 
##              0              0              0              1              0 
##            sex            bmi            ahi            ess           mmse 
##              0              0              0              0              0 
## avg_daily_cpap      adherence        odsi_bl        odsi_6m       adcs_12m 
##              0              0              0             21             45
\end{verbatim}

\begin{Shaded}
\begin{Highlighting}[]
\DocumentationTok{\#\# {-}{-}{-} Summary of all variables {-}{-}{-}}
\FunctionTok{summary}\NormalTok{(Data)}
\end{Highlighting}
\end{Shaded}

\begin{verbatim}
##   subject_id                         ethnicity            education  
##  Length:174         Hispanic or Latino    : 13   <= high school: 36  
##  Class :character   Not Hispanic or Latino:161   > high school :138  
##  Mode  :character                                                    
##                                                                      
##                                                                      
##                                                                      
##                                                                      
##     race          age            sex                 bmi       
##  White:124   Min.   :55.00   Length:174         Min.   :20.00  
##  Black: 37   1st Qu.:61.00   Class :character   1st Qu.:37.50  
##  Other: 12   Median :66.50   Mode  :character   Median :42.32  
##  NA's :  1   Mean   :66.86                      Mean   :42.18  
##              3rd Qu.:72.00                      3rd Qu.:46.88  
##              Max.   :85.00                      Max.   :65.11  
##                                                                
##       ahi              ess              mmse       avg_daily_cpap 
##  Min.   : 15.00   Min.   : 0.000   Min.   :21.00   Min.   :0.000  
##  1st Qu.: 19.50   1st Qu.: 6.000   1st Qu.:26.25   1st Qu.:3.800  
##  Median : 28.45   Median : 8.000   Median :28.00   Median :5.667  
##  Mean   : 34.78   Mean   : 8.885   Mean   :27.60   Mean   :5.151  
##  3rd Qu.: 44.62   3rd Qu.:12.000   3rd Qu.:29.00   3rd Qu.:7.046  
##  Max.   :119.40   Max.   :22.000   Max.   :30.00   Max.   :9.300  
##                                                                   
##         adherence      odsi_bl          odsi_6m          adcs_12m    
##  Non-adherent: 46   Min.   : 0.000   Min.   : 0.000   Min.   :1.000  
##  Adherent    :128   1st Qu.: 2.000   1st Qu.: 2.000   1st Qu.:2.000  
##                     Median : 6.000   Median : 3.000   Median :3.000  
##                     Mean   : 7.983   Mean   : 5.268   Mean   :3.194  
##                     3rd Qu.:13.000   3rd Qu.: 8.000   3rd Qu.:4.000  
##                     Max.   :22.000   Max.   :18.000   Max.   :6.000  
##                                      NA's   :21       NA's   :45
\end{verbatim}

\begin{Shaded}
\begin{Highlighting}[]
\DocumentationTok{\#\# Save cleaned dataset}
\FunctionTok{write.csv}\NormalTok{(Data, }\StringTok{"CPAPAdherence\_Data\_Clean\_Ready.csv"}\NormalTok{, }\AttributeTok{row.names =} \ConstantTok{FALSE}\NormalTok{)}

\FunctionTok{library}\NormalTok{(gtsummary)}
\FunctionTok{library}\NormalTok{(dplyr)}
\FunctionTok{library}\NormalTok{(broom)}

\FunctionTok{names}\NormalTok{(Data)}
\end{Highlighting}
\end{Shaded}

\begin{verbatim}
##  [1] "subject_id"     "ethnicity"      "education"      "race"          
##  [5] "age"            "sex"            "bmi"            "ahi"           
##  [9] "ess"            "mmse"           "avg_daily_cpap" "adherence"     
## [13] "odsi_bl"        "odsi_6m"        "adcs_12m"
\end{verbatim}

\begin{Shaded}
\begin{Highlighting}[]
\NormalTok{Data }\OtherTok{\textless{}{-}}\NormalTok{ Data }\SpecialCharTok{\%\textgreater{}\%}
  \FunctionTok{rename}\NormalTok{(}
    \AttributeTok{Ethnicity =}\NormalTok{ ethnicity,}
    \AttributeTok{Education =}\NormalTok{ education,}
    \AttributeTok{Race =}\NormalTok{ race,}
    \AttributeTok{Age =}\NormalTok{ age,}
    \AttributeTok{Sex =}\NormalTok{ sex,}
    \AttributeTok{BMI =}\NormalTok{ bmi,}
    \AttributeTok{AHI =}\NormalTok{ ahi,}
    \AttributeTok{ESS =}\NormalTok{ ess,}
    \AttributeTok{MMSE =}\NormalTok{ mmse,}
    \AttributeTok{ODSI\_baseline =}\NormalTok{ odsi\_bl,}
    \AttributeTok{ADCS\_MCI\_12m =}\NormalTok{ adcs\_12m,}
    \AttributeTok{ODSI\_6m =}\NormalTok{ odsi\_6m,}
    \StringTok{\textasciigrave{}}\AttributeTok{Average daily CPAP (hr/night)}\StringTok{\textasciigrave{}} \OtherTok{=}\NormalTok{ avg\_daily\_cpap}
\NormalTok{  )}
  
  
\NormalTok{Data}\SpecialCharTok{$}\NormalTok{adherence }\OtherTok{\textless{}{-}} \FunctionTok{factor}\NormalTok{(Data}\SpecialCharTok{$}\NormalTok{adherence,}
                                 \AttributeTok{levels =} \FunctionTok{c}\NormalTok{(}\StringTok{"Non{-}adherent"}\NormalTok{, }\StringTok{"Adherent"}\NormalTok{))}
\FunctionTok{library}\NormalTok{(dplyr)                                }
\NormalTok{table1 }\OtherTok{\textless{}{-}}\NormalTok{ Data }\SpecialCharTok{\%\textgreater{}\%}
  \FunctionTok{select}\NormalTok{(Ethnicity, Education, Race, Age, Sex, BMI, AHI, ESS, MMSE,}
         \StringTok{\textasciigrave{}}\AttributeTok{Average daily CPAP (hr/night)}\StringTok{\textasciigrave{}}\NormalTok{, adherence, ODSI\_baseline, ODSI\_6m, }
\NormalTok{         ADCS\_MCI\_12m) }\SpecialCharTok{\%\textgreater{}\%}
  \FunctionTok{tbl\_summary}\NormalTok{(}
    \AttributeTok{by =}\NormalTok{ adherence,}
    \AttributeTok{statistic =} \FunctionTok{list}\NormalTok{(}
      \FunctionTok{all\_continuous}\NormalTok{() }\SpecialCharTok{\textasciitilde{}} \StringTok{"\{mean\} ± \{sd\}"}\NormalTok{,}
      \FunctionTok{all\_categorical}\NormalTok{() }\SpecialCharTok{\textasciitilde{}} \StringTok{"\{n\} (\{p\}\%)"}
\NormalTok{    ),}
    \AttributeTok{digits =} \FunctionTok{all\_continuous}\NormalTok{() }\SpecialCharTok{\textasciitilde{}} \DecValTok{2}
\NormalTok{  ) }\SpecialCharTok{\%\textgreater{}\%}
  \FunctionTok{add\_p}\NormalTok{(}\AttributeTok{test =} \FunctionTok{list}\NormalTok{(}
    \FunctionTok{all\_continuous}\NormalTok{() }\SpecialCharTok{\textasciitilde{}} \StringTok{"t.test"}\NormalTok{,}
    \FunctionTok{all\_categorical}\NormalTok{() }\SpecialCharTok{\textasciitilde{}} \StringTok{"chisq.test"}
\NormalTok{  )) }\SpecialCharTok{\%\textgreater{}\%}
  \FunctionTok{add\_n}\NormalTok{() }\SpecialCharTok{\%\textgreater{}\%}
  \FunctionTok{modify\_header}\NormalTok{(label }\SpecialCharTok{\textasciitilde{}} \StringTok{"**Characteristic**"}\NormalTok{) }\SpecialCharTok{\%\textgreater{}\%}
  \FunctionTok{modify\_caption}\NormalTok{(}\StringTok{"Demographic and Clinical Characteristics (n = 174)"}\NormalTok{)}
\end{Highlighting}
\end{Shaded}

\begin{verbatim}
## The following warnings were returned during `modify_caption()`:
\end{verbatim}

\begin{verbatim}
## ! For variable `ADCS_MCI_12m` (`adherence`) and "statistic", "p.value", and
##   "parameter" statistics: Chi-squared approximation may be incorrect
## ! For variable `Ethnicity` (`adherence`) and "statistic", "p.value", and
##   "parameter" statistics: Chi-squared approximation may be incorrect
## ! For variable `MMSE` (`adherence`) and "statistic", "p.value", and "parameter"
##   statistics: Chi-squared approximation may be incorrect
## ! For variable `Race` (`adherence`) and "statistic", "p.value", and "parameter"
##   statistics: Chi-squared approximation may be incorrect
\end{verbatim}

\begin{Shaded}
\begin{Highlighting}[]
\NormalTok{vars }\OtherTok{\textless{}{-}} \FunctionTok{c}\NormalTok{(}\StringTok{"ethnicity"}\NormalTok{, }\StringTok{"education"}\NormalTok{, }\StringTok{"race"}\NormalTok{, }\StringTok{"age"}\NormalTok{, }\StringTok{"sex"}\NormalTok{, }\StringTok{"BMI"}\NormalTok{, }\StringTok{"AHI"}\NormalTok{,}
          \StringTok{"ESS"}\NormalTok{, }\StringTok{"MMSE"}\NormalTok{,}\StringTok{"Average daily CPAP (hr/night)"}\NormalTok{, }\StringTok{"ODSI\_baseline"}\NormalTok{, }
          \StringTok{"ODSI\_6m"}\NormalTok{, }\StringTok{"ADCS\_MCI\_12m"}\NormalTok{)}
\FunctionTok{library}\NormalTok{(tableone)}
\NormalTok{tab1 }\OtherTok{\textless{}{-}} \FunctionTok{CreateTableOne}\NormalTok{(}\AttributeTok{vars =}\NormalTok{ vars, }\AttributeTok{strata =} \StringTok{"adherence"}\NormalTok{, }\AttributeTok{data =}\NormalTok{ Data)}
\end{Highlighting}
\end{Shaded}

\begin{verbatim}
## Warning in ModuleReturnVarsExist(vars, data): The data frame does not have:
## ethnicity education race age sex Dropped
\end{verbatim}

\begin{Shaded}
\begin{Highlighting}[]
\FunctionTok{print}\NormalTok{(tab1, }\AttributeTok{showAllLevels =} \ConstantTok{TRUE}\NormalTok{, }\AttributeTok{smd =} \ConstantTok{TRUE}\NormalTok{)}
\end{Highlighting}
\end{Shaded}

\begin{verbatim}
##                                            Stratified by adherence
##                                             level Non-adherent  Adherent     
##   n                                                  46           128        
##   BMI (mean (SD))                                 42.15 (7.37)  42.20 (7.18) 
##   AHI (mean (SD))                                 35.59 (19.91) 34.49 (21.20)
##   ESS (mean (SD))                                  9.02 (4.79)   8.84 (5.04) 
##   MMSE (mean (SD))                                27.39 (1.81)  27.67 (1.77) 
##   Average daily CPAP (hr/night) (mean (SD))        1.61 (1.35)   6.42 (1.32) 
##   ODSI_baseline (mean (SD))                        8.30 (5.77)   7.87 (6.21) 
##   ODSI_6m (mean (SD))                              6.18 (5.35)   5.01 (4.80) 
##   ADCS_MCI_12m (mean (SD))                         3.69 (1.41)   3.07 (1.47) 
##                                            Stratified by adherence
##                                             p      test SMD   
##   n                                                           
##   BMI (mean (SD))                            0.966       0.007
##   AHI (mean (SD))                            0.758       0.054
##   ESS (mean (SD))                            0.828       0.038
##   MMSE (mean (SD))                           0.361       0.157
##   Average daily CPAP (hr/night) (mean (SD)) <0.001       3.613
##   ODSI_baseline (mean (SD))                  0.677       0.073
##   ODSI_6m (mean (SD))                        0.224       0.230
##   ADCS_MCI_12m (mean (SD))                   0.053       0.434
\end{verbatim}

\begin{Shaded}
\begin{Highlighting}[]
\FunctionTok{write.csv}\NormalTok{(}\FunctionTok{print}\NormalTok{(tab1, }\AttributeTok{showAllLevels =} \ConstantTok{TRUE}\NormalTok{, }\AttributeTok{smd =} \ConstantTok{TRUE}\NormalTok{),}
          \StringTok{"Table1\_Summary.csv"}\NormalTok{, }\AttributeTok{row.names =} \ConstantTok{FALSE}\NormalTok{)}
\end{Highlighting}
\end{Shaded}

\begin{verbatim}
##                                            Stratified by adherence
##                                             level Non-adherent  Adherent     
##   n                                                  46           128        
##   BMI (mean (SD))                                 42.15 (7.37)  42.20 (7.18) 
##   AHI (mean (SD))                                 35.59 (19.91) 34.49 (21.20)
##   ESS (mean (SD))                                  9.02 (4.79)   8.84 (5.04) 
##   MMSE (mean (SD))                                27.39 (1.81)  27.67 (1.77) 
##   Average daily CPAP (hr/night) (mean (SD))        1.61 (1.35)   6.42 (1.32) 
##   ODSI_baseline (mean (SD))                        8.30 (5.77)   7.87 (6.21) 
##   ODSI_6m (mean (SD))                              6.18 (5.35)   5.01 (4.80) 
##   ADCS_MCI_12m (mean (SD))                         3.69 (1.41)   3.07 (1.47) 
##                                            Stratified by adherence
##                                             p      test SMD   
##   n                                                           
##   BMI (mean (SD))                            0.966       0.007
##   AHI (mean (SD))                            0.758       0.054
##   ESS (mean (SD))                            0.828       0.038
##   MMSE (mean (SD))                           0.361       0.157
##   Average daily CPAP (hr/night) (mean (SD)) <0.001       3.613
##   ODSI_baseline (mean (SD))                  0.677       0.073
##   ODSI_6m (mean (SD))                        0.224       0.230
##   ADCS_MCI_12m (mean (SD))                   0.053       0.434
\end{verbatim}

\begin{Shaded}
\begin{Highlighting}[]
\NormalTok{table1     }\CommentTok{\#}
\end{Highlighting}
\end{Shaded}

\begin{table}[t]
\caption{\label{tab:table 1}Demographic and Clinical Characteristics (n = 174)} 
\fontsize{12.0pt}{14.0pt}\selectfont
\begin{tabular*}{\linewidth}{@{\extracolsep{\fill}}lcccc}
\toprule
\textbf{Characteristic} & \textbf{N} & \textbf{Non-adherent}  N = 46\textsuperscript{\textit{1}} & \textbf{Adherent}  N = 128\textsuperscript{\textit{1}} & \textbf{p-value}\textsuperscript{\textit{2}} \\ 
\midrule\addlinespace[2.5pt]
Ethnicity & 174 &  &  & >0.9 \\ 
    Hispanic or Latino &  & 3 (6.5\%) & 10 (7.8\%) &  \\ 
    Not Hispanic or Latino &  & 43 (93\%) & 118 (92\%) &  \\ 
Education & 174 &  &  & 0.4 \\ 
    <= high school &  & 12 (26\%) & 24 (19\%) &  \\ 
    > high school &  & 34 (74\%) & 104 (81\%) &  \\ 
Race & 173 &  &  & <0.001 \\ 
    White &  & 23 (51\%) & 101 (79\%) &  \\ 
    Black &  & 19 (42\%) & 18 (14\%) &  \\ 
    Other &  & 3 (6.7\%) & 9 (7.0\%) &  \\ 
    Unknown &  & 1 & 0 &  \\ 
Age & 174 & 66.98 \ensuremath{\pm} 7.57 & 66.81 \ensuremath{\pm} 7.53 & 0.9 \\ 
Sex & 174 &  &  & >0.9 \\ 
    Female &  & 22 (48\%) & 58 (45\%) &  \\ 
    Male &  & 24 (52\%) & 70 (55\%) &  \\ 
BMI & 174 & 42.15 \ensuremath{\pm} 7.37 & 42.20 \ensuremath{\pm} 7.18 & >0.9 \\ 
AHI & 174 & 35.59 \ensuremath{\pm} 19.91 & 34.49 \ensuremath{\pm} 21.20 & 0.8 \\ 
ESS & 174 & 9.02 \ensuremath{\pm} 4.79 & 8.84 \ensuremath{\pm} 5.04 & 0.8 \\ 
MMSE & 174 &  &  & 0.6 \\ 
    21 &  & 1 (2.2\%) & 0 (0\%) &  \\ 
    23 &  & 0 (0\%) & 3 (2.3\%) &  \\ 
    24 &  & 2 (4.3\%) & 4 (3.1\%) &  \\ 
    25 &  & 3 (6.5\%) & 10 (7.8\%) &  \\ 
    26 &  & 6 (13\%) & 15 (12\%) &  \\ 
    27 &  & 9 (20\%) & 20 (16\%) &  \\ 
    28 &  & 11 (24\%) & 24 (19\%) &  \\ 
    29 &  & 11 (24\%) & 35 (27\%) &  \\ 
    30 &  & 3 (6.5\%) & 17 (13\%) &  \\ 
Average daily CPAP (hr/night) & 174 & 1.61 \ensuremath{\pm} 1.35 & 6.42 \ensuremath{\pm} 1.32 & <0.001 \\ 
ODSI\_baseline & 174 & 8.30 \ensuremath{\pm} 5.77 & 7.87 \ensuremath{\pm} 6.21 & 0.7 \\ 
ODSI\_6m & 153 & 6.18 \ensuremath{\pm} 5.35 & 5.01 \ensuremath{\pm} 4.80 & 0.3 \\ 
    Unknown &  & 12 & 9 &  \\ 
ADCS\_MCI\_12m & 129 &  &  & 0.5 \\ 
    1 &  & 2 (7.7\%) & 17 (17\%) &  \\ 
    2 &  & 3 (12\%) & 25 (24\%) &  \\ 
    3 &  & 6 (23\%) & 19 (18\%) &  \\ 
    4 &  & 8 (31\%) & 25 (24\%) &  \\ 
    5 &  & 4 (15\%) & 10 (9.7\%) &  \\ 
    6 &  & 3 (12\%) & 7 (6.8\%) &  \\ 
    Unknown &  & 20 & 25 &  \\ 
\bottomrule
\end{tabular*}
\begin{minipage}{\linewidth}
\textsuperscript{\textit{1}}n (\%); Mean ± SD\\
\textsuperscript{\textit{2}}Pearson's Chi-squared test; Welch Two Sample t-test\\
\end{minipage}
\end{table}

\subsection{1B.Choose a single characteristic with a significant p-value
when comparing between adherence and non-adherent groups, and describe
in 2-3 sentences what this means in plain
English.}\label{b.choose-a-single-characteristic-with-a-significant-p-value-when-comparing-between-adherence-and-non-adherent-groups-and-describe-in-2-3-sentences-what-this-means-in-plain-english.}

The adherent group's average daily CPAP usage was considerably higher
(approximately 6.4 hours per night) than that of the non-adherent group
(approximately 1.6 hours per night, p \textless{} 0.001).

In plain English, adherent participants utilized their CPAP machines for
significantly extended periods of time each night. This affirms that the
adherence status is indicative of the actual treatment behavior and
emphasizes that device utilization is the primary determining factor
between the 2 groups.

\subsection{1C.Choose a single characteristic with a non-significant
p-value when comparing between adherence and non-adherent groups, and
describe in 2-3 sentences what this means in plain
English.}\label{c.choose-a-single-characteristic-with-a-non-significant-p-value-when-comparing-between-adherence-and-non-adherent-groups-and-describe-in-2-3-sentences-what-this-means-in-plain-english.}

There was no significant difference in age between adherent and
non-adherent participants (p ≈ 0.90).

This implies that the adherence to CPAP was not influenced by the age of
the participants; younger and older individuals were equally likely to
be adherent or non-adherent. In layman's terms, the frequency with which
an individual employs their CPAP device does not seem to be influenced
by their age.

\subsection{2.Test the null hypothesis that there is no difference in
change from baseline to 6 months for ODSI for adherent versus
non-adherent participants. Write out each step of the hypothesis test
and clearly interpret your results in plain English in 2-3
sentences.}\label{test-the-null-hypothesis-that-there-is-no-difference-in-change-from-baseline-to-6-months-for-odsi-for-adherent-versus-non-adherent-participants.-write-out-each-step-of-the-hypothesis-test-and-clearly-interpret-your-results-in-plain-english-in-2-3-sentences.}

Null Hypothesis (H₀): There is no difference in the change of ODSI from
baseline to 6 months between adherent and non-adherent participants.

\[
H_0: \mu_{\text{adherent}} = \mu_{\text{non-adherent}}
\]

Alternative Hypothesis: \# Alternative Hypothesis (H₁): There is a
difference in the change of ODSI between adherent and non-adherent
participants.

\[
H_1: \mu_{\text{adherent}} \neq \mu_{\text{non-adherent}}
\]

\begin{Shaded}
\begin{Highlighting}[]
\DocumentationTok{\#\# Remove rows with missing values in \textquotesingle{}odsi\_6m\textquotesingle{}}
\NormalTok{clean\_data }\OtherTok{\textless{}{-}}\NormalTok{ Data }\SpecialCharTok{\%\textgreater{}\%}
  \FunctionTok{filter}\NormalTok{(}\SpecialCharTok{!}\FunctionTok{is.na}\NormalTok{(ODSI\_6m))}

\DocumentationTok{\#\# Check the structure of the cleaned data}
\FunctionTok{str}\NormalTok{(clean\_data)}
\end{Highlighting}
\end{Shaded}

\begin{verbatim}
## tibble [153 x 15] (S3: tbl_df/tbl/data.frame)
##  $ subject_id                   : chr [1:153] "11-01102" "11-01153" "11-01442" "11-01634" ...
##  $ Ethnicity                    : Factor w/ 2 levels "Hispanic or Latino",..: 2 2 2 2 2 2 2 2 2 2 ...
##  $ Education                    : Factor w/ 2 levels "<= high school",..: 2 2 2 2 2 2 1 2 2 2 ...
##  $ Race                         : Factor w/ 3 levels "White","Black",..: 2 1 1 1 2 1 2 1 1 2 ...
##  $ Age                          : num [1:153] 62 71 75 62 55 70 67 75 75 63 ...
##  $ Sex                          : chr [1:153] "Female" "Male" "Female" "Male" ...
##  $ BMI                          : num [1:153] 51 42.8 53.5 35.8 41.4 ...
##  $ AHI                          : num [1:153] 22.4 24.4 19.9 21.5 18.2 33.4 15.3 22 79.7 37 ...
##  $ ESS                          : num [1:153] 17 6 4 9 7 8 1 5 6 22 ...
##  $ MMSE                         : num [1:153] 27 30 30 29 29 26 29 29 30 26 ...
##  $ Average daily CPAP (hr/night): num [1:153] 6.45 9.05 4.57 7.62 6.3 ...
##  $ adherence                    : Factor w/ 2 levels "Non-adherent",..: 2 2 2 2 2 2 1 2 2 2 ...
##  $ ODSI_baseline                : num [1:153] 5 0 2 16 10 2 0 0 4 19 ...
##  $ ODSI_6m                      : num [1:153] 4 0 2 0 8 2 0 1 3 18 ...
##  $ ADCS_MCI_12m                 : num [1:153] 4 2 4 1 1 2 4 1 1 6 ...
\end{verbatim}

\begin{Shaded}
\begin{Highlighting}[]
\DocumentationTok{\#\# Calculate the change in ODSI score from baseline to 6 months}
\NormalTok{clean\_data}\SpecialCharTok{$}\NormalTok{odsi\_change }\OtherTok{\textless{}{-}}\NormalTok{ clean\_data}\SpecialCharTok{$}\NormalTok{ODSI\_6m }\SpecialCharTok{{-}}\NormalTok{ clean\_data}\SpecialCharTok{$}\NormalTok{ODSI\_baseline}

\DocumentationTok{\#\# Impute missing values with the median of \textquotesingle{}odsi\_6m\textquotesingle{}}
\NormalTok{Data}\SpecialCharTok{$}\NormalTok{ODSI\_6m[}\FunctionTok{is.na}\NormalTok{(Data}\SpecialCharTok{$}\NormalTok{ODSI\_6m)] }\OtherTok{\textless{}{-}} \FunctionTok{median}\NormalTok{(Data}\SpecialCharTok{$}\NormalTok{ODSI\_6m, }\AttributeTok{na.rm =} \ConstantTok{TRUE}\NormalTok{)}
\FunctionTok{summary}\NormalTok{(Data}\SpecialCharTok{$}\NormalTok{ODSI\_6m)}
\end{Highlighting}
\end{Shaded}

\begin{verbatim}
##    Min. 1st Qu.  Median    Mean 3rd Qu.    Max. 
##   0.000   2.000   3.000   4.994   8.000  18.000
\end{verbatim}

\begin{Shaded}
\begin{Highlighting}[]
\DocumentationTok{\#\# Calculate ODSI change again }
\NormalTok{clean\_data }\OtherTok{\textless{}{-}}\NormalTok{ clean\_data }\SpecialCharTok{\%\textgreater{}\%}
  \FunctionTok{mutate}\NormalTok{(}\AttributeTok{odsi\_change =}\NormalTok{ ODSI\_6m }\SpecialCharTok{{-}}\NormalTok{ ODSI\_baseline)}
  
\DocumentationTok{\#\# Test normality}
\FunctionTok{shapiro.test}\NormalTok{(clean\_data}\SpecialCharTok{$}\NormalTok{odsi\_change)}
\end{Highlighting}
\end{Shaded}

\begin{verbatim}
## 
##  Shapiro-Wilk normality test
## 
## data:  clean_data$odsi_change
## W = 0.95872, p-value = 0.0001592
\end{verbatim}

\begin{Shaded}
\begin{Highlighting}[]
\FunctionTok{wilcox.test}\NormalTok{(odsi\_change }\SpecialCharTok{\textasciitilde{}}\NormalTok{ adherence, }\AttributeTok{data =}\NormalTok{ clean\_data, }
            \AttributeTok{alternative =} \StringTok{"two.sided"}\NormalTok{)}
\end{Highlighting}
\end{Shaded}

\begin{verbatim}
## 
##  Wilcoxon rank sum test with continuity correction
## 
## data:  odsi_change by adherence
## W = 2282.5, p-value = 0.254
## alternative hypothesis: true location shift is not equal to 0
\end{verbatim}

Since the p-value is less than 0.05, we reject the null hypothesis of
normality. This indicates that the odsi\_change data are not normally
distributed.

Wilcoxon Rank-Sum Test (Mann-Whitney U test): The p-value from the
Mann-Whitney U test is greater than 0.05, so we fail to reject the null
hypothesis. This means there is no significant difference in the change
of ODSI scores from baseline to 6 months between adherent and
non-adherent participants.

\subsection{3. Hypothesis Test: Probability of Excessive Daytime
Sleepiness at Baseline vs.~6
Months}\label{hypothesis-test-probability-of-excessive-daytime-sleepiness-at-baseline-vs.-6-months}

The objective is to ascertain if the likelihood of excessive daytime
drowsiness, as determined by a cutoff score of 6 or above on the ODSI
(Observation and Interview Based Diurnal drowsiness Inventory), is
consistent from baseline to 6 months.

State the Hypotheses Null Hypothesis (H₀): The likelihood of excessive
daytime drowsiness remains constant from baseline to the 6-month mark.

\[
H_0: P(\text{ODSI} \geq 6 \text{ at baseline}) = P(\text{ODSI} \geq 6 \text{ at 6 months})
\]

Alternative Hypothesis (H₁): The probability of excessive daytime
sleepiness is different at baseline compared to at 6 months.

\[
H_1: P(\text{ODSI} \geq 6 \text{ at baseline}) \neq P(\text{ODSI} \geq 6 \text{ at 6 months})
\] Pick the Right TesTest: This is a comparison of the probabilities
(proportions) of two groups: baseline and 6 months. Because we are
working with categorical data (whether or not someone is excessively
sleepy), a McNemar's test or a Chi-squared test would work. However, we
will utilize the McNemar's test for paired categorical data (baseline
vs.~6 months for the same people).

Get the Data Ready: We need to make a variable that shows if someone has
excessive daytime sleepiness (ODSI \textgreater{} 6) at the start and
again at 6 months.

Make a variable with two values: 1 for too much sleepiness during the
day (ODSI ≥ 6) 0 for not being too sleepy during the day (ODSI
\textless{} 6)

\begin{Shaded}
\begin{Highlighting}[]
\CommentTok{\# Create a binary variable for ODSI ≥ 6 at baseline and 6 months}
\NormalTok{Data}\SpecialCharTok{$}\NormalTok{baseline\_sleepy }\OtherTok{\textless{}{-}} \FunctionTok{ifelse}\NormalTok{(Data}\SpecialCharTok{$}\NormalTok{ODSI\_baseline }\SpecialCharTok{\textgreater{}=} \DecValTok{6}\NormalTok{, }\DecValTok{1}\NormalTok{, }\DecValTok{0}\NormalTok{)}
\NormalTok{Data}\SpecialCharTok{$}\NormalTok{month6\_sleepy }\OtherTok{\textless{}{-}} \FunctionTok{ifelse}\NormalTok{(Data}\SpecialCharTok{$}\NormalTok{ODSI\_6m }\SpecialCharTok{\textgreater{}=} \DecValTok{6}\NormalTok{, }\DecValTok{1}\NormalTok{, }\DecValTok{0}\NormalTok{)}

\DocumentationTok{\#\# Create a contingency table for McNemar\textquotesingle{}s test}
\NormalTok{table\_sleepiness }\OtherTok{\textless{}{-}} \FunctionTok{table}\NormalTok{(Data}\SpecialCharTok{$}\NormalTok{baseline\_sleepy, Data}\SpecialCharTok{$}\NormalTok{month6\_sleepy)}

\DocumentationTok{\#\# Perform McNemar\textquotesingle{}s test}
\NormalTok{mcnemar\_test }\OtherTok{\textless{}{-}} \FunctionTok{mcnemar.test}\NormalTok{(table\_sleepiness)}
\NormalTok{mcnemar\_test}
\end{Highlighting}
\end{Shaded}

\begin{verbatim}
## 
##  McNemar's Chi-squared test with continuity correction
## 
## data:  table_sleepiness
## McNemar's chi-squared = 14.561, df = 1, p-value = 0.0001357
\end{verbatim}

Conclusion: Reject the null hypothesis because the p-value (0.0001357)
is less than 0.05. There is a statistically significant difference in
the likelihood of excessive daytime sleepiness (ODSI \textgreater{} 6)
between the baseline and 6 months. To put it another way, the chances of
experiencing too much daytime sleepiness at the start are not the same
as the chances at 6 months. This indicates that the variation in
tiredness across time is substantial in the sample.

\subsection{4A.Test the Null Hypothesis with Known Population Standard
Deviation
1.}\label{a.test-the-null-hypothesis-with-known-population-standard-deviation-1.}

Null Hypothesis (H₀): The average MMSE score for all study participants
is over 23, which means they are not having any cognitive problems.

Alternative Hypothesis (H₁): The mean MMSE score for all study
participants is less than or equal to 23, signifying cognitive
impairment.

\[
H_0: \mu_{\text{study participants}} \geq 23
\] \[
H_1: \mu_{\text{study participants}} \leq 23
\]

\begin{Shaded}
\begin{Highlighting}[]
\DocumentationTok{\#\# Sample data check }

\NormalTok{sample\_mean }\OtherTok{\textless{}{-}} \FunctionTok{mean}\NormalTok{(Data}\SpecialCharTok{$}\NormalTok{MMSE, }\AttributeTok{na.rm =} \ConstantTok{TRUE}\NormalTok{)  }\CommentTok{\# Calculate the sample mean}
\NormalTok{population\_mean }\OtherTok{\textless{}{-}} \DecValTok{23}  \CommentTok{\# The threshold for cognitive impairment}
\NormalTok{population\_sd }\OtherTok{\textless{}{-}} \FloatTok{2.0}  \CommentTok{\# Known population standard deviation}
\NormalTok{n }\OtherTok{\textless{}{-}} \FunctionTok{length}\NormalTok{(Data}\SpecialCharTok{$}\NormalTok{MMSE)  }\CommentTok{\# Sample size}

\DocumentationTok{\#\# Calculate the Z statistic}

\NormalTok{Z }\OtherTok{\textless{}{-}}\NormalTok{ (sample\_mean }\SpecialCharTok{{-}}\NormalTok{ population\_mean) }\SpecialCharTok{/}\NormalTok{ (population\_sd }\SpecialCharTok{/} \FunctionTok{sqrt}\NormalTok{(n))}

\DocumentationTok{\#\# Calculate p{-}value for the one{-}tailed test (lower{-}tailed)}

\NormalTok{p\_value }\OtherTok{\textless{}{-}} \FunctionTok{pnorm}\NormalTok{(Z)  }

\DocumentationTok{\#\# Print results}

\NormalTok{Z}
\end{Highlighting}
\end{Shaded}

\begin{verbatim}
## [1] 30.32392
\end{verbatim}

\begin{Shaded}
\begin{Highlighting}[]
\NormalTok{p\_value}
\end{Highlighting}
\end{Shaded}

\begin{verbatim}
## [1] 1
\end{verbatim}

Understand the Z Statistic: very high number, 30.32, is the Z statistic.
This means that the sample mean MMSE is far higher than the criteria for
cognitive impairment, which is 23.

p-value: The p-value of 1 means that it is very unlikely that the sample
mean MMSE is less than 23. The p-value being more than 0.05 means that
there isn't enough evidence to reject the null hypothesis. To put it
another way: This indicates that there is no substantial difference
between the sample mean and the threshold value of 23, hence supporting
the null hypothesis.

Conclusion for non-statistical people: We do not reject the null
hypothesis because the p-value is quite high (1). This indicates that
the study participants possess a mean MMSE score exceeding 23, with no
substantial evidence of cognitive impairment among the sample.

\subsection{4B.Test the Null Hypothesis with an Unknown Population
Standard
Deviation}\label{b.test-the-null-hypothesis-with-an-unknown-population-standard-deviation}

Null Hypothesis (H₀): The average MMSE score of the study participants
is 23 or higher, which means there is no cognitive impairment.

Alternative Hypothesis (H₁): The average MMSE score of the people in the
study is 23 or lower, which means they have cognitive problems.

\[
H_0: \mu_{\text{study participants}} \geq 23
\]

\[
H_1: \mu_{\text{study participants}} \leq 23
\]

\begin{Shaded}
\begin{Highlighting}[]
\DocumentationTok{\#\# Sample data check: }

\NormalTok{sample\_mean }\OtherTok{\textless{}{-}} \FunctionTok{mean}\NormalTok{(Data}\SpecialCharTok{$}\NormalTok{MMSE, }\AttributeTok{na.rm =} \ConstantTok{TRUE}\NormalTok{)  }\CommentTok{\# Calculate the sample mean}
\NormalTok{population\_mean }\OtherTok{\textless{}{-}} \DecValTok{23}  \CommentTok{\# The threshold for cognitive impairment}
\NormalTok{sample\_sd }\OtherTok{\textless{}{-}} \FunctionTok{sd}\NormalTok{(Data}\SpecialCharTok{$}\NormalTok{MMSE, }\AttributeTok{na.rm =} \ConstantTok{TRUE}\NormalTok{)  }\CommentTok{\# Sample standard deviation }
\NormalTok{n }\OtherTok{\textless{}{-}} \FunctionTok{length}\NormalTok{(Data}\SpecialCharTok{$}\NormalTok{MMSE)  }\CommentTok{\# Sample size}

\DocumentationTok{\#\# Calculate the t{-}statistic}

\NormalTok{t\_stat }\OtherTok{\textless{}{-}}\NormalTok{ (sample\_mean }\SpecialCharTok{{-}}\NormalTok{ population\_mean) }\SpecialCharTok{/}\NormalTok{ (sample\_sd }\SpecialCharTok{/} \FunctionTok{sqrt}\NormalTok{(n))}

\DocumentationTok{\#\# Degrees of freedom for the t{-}distribution}

\NormalTok{df }\OtherTok{\textless{}{-}}\NormalTok{ n }\SpecialCharTok{{-}} \DecValTok{1}  \CommentTok{\# Degrees of freedom}

\DocumentationTok{\#\# Calculate p{-}value for the one{-}tailed test}

\NormalTok{p\_value\_t }\OtherTok{\textless{}{-}} \FunctionTok{pt}\NormalTok{(t\_stat, df)  }\CommentTok{\# For one{-}tailed lower test}

\DocumentationTok{\#\# Print results}
\NormalTok{t\_stat}
\end{Highlighting}
\end{Shaded}

\begin{verbatim}
## [1] 34.08097
\end{verbatim}

\begin{Shaded}
\begin{Highlighting}[]
\NormalTok{p\_value\_t}
\end{Highlighting}
\end{Shaded}

\begin{verbatim}
## [1] 1
\end{verbatim}

Understanding the t-statistic: The t-statistic of 34.08 is exceptionally
high, which means that the sample mean MMSE is far higher than the
criterion for cognitive impairment, which is 23. This shows that the
sample mean and the hypothesized population mean are very different from
each other.

Understanding the p-value: The p-value of 1 is exceptionally high. This
means that the sample mean MMSE is very unlikely to be less than 23.
This means that there is no proof to reject the null hypothesis.

We can't reject the null hypothesis because the p-value is higher than
0.05.

Conclusion for an Audience Without a Statistics Background: The p-value
is 1, which suggests that the participants' average MMSE score is much
higher than 23. This means that there is no proof that the individuals
are cognitively impaired based on the MMSE score. In other words, we
can't say for sure that the people in this study have cognitive problems
solely based on the MMSE results.

\subsection{5.Compute and Report the Mean and Standard Deviation for the
Variable Representing Average Daily CPAP
Use}\label{compute-and-report-the-mean-and-standard-deviation-for-the-variable-representing-average-daily-cpap-use}

\begin{Shaded}
\begin{Highlighting}[]
\DocumentationTok{\#\# 5A Calculate the mean and standard deviation for \textquotesingle{}avg\_daily\_cpap\textquotesingle{}}
\NormalTok{mean\_cpap }\OtherTok{\textless{}{-}} \FunctionTok{mean}\NormalTok{(Data}\SpecialCharTok{$}\StringTok{\textasciigrave{}}\AttributeTok{Average daily CPAP (hr/night)}\StringTok{\textasciigrave{}}\NormalTok{, }\AttributeTok{na.rm =} \ConstantTok{TRUE}\NormalTok{)}
\NormalTok{sd\_cpap }\OtherTok{\textless{}{-}} \FunctionTok{sd}\NormalTok{(Data}\SpecialCharTok{$}\StringTok{\textasciigrave{}}\AttributeTok{Average daily CPAP (hr/night)}\StringTok{\textasciigrave{}}\NormalTok{, }\AttributeTok{na.rm =} \ConstantTok{TRUE}\NormalTok{)}

\DocumentationTok{\#\# Report the results}
\NormalTok{mean\_cpap}
\end{Highlighting}
\end{Shaded}

\begin{verbatim}
## [1] 5.150766
\end{verbatim}

\begin{Shaded}
\begin{Highlighting}[]
\NormalTok{sd\_cpap}
\end{Highlighting}
\end{Shaded}

\begin{verbatim}
## [1] 2.504029
\end{verbatim}

\#Interpretation: The average daily CPAP use in this sample is
approximately 5.15 hours per night. The standard deviation is 2.50
hours, indicating that there is a fairly wide variation in CPAP usage
among participants. This suggests that while some individuals use CPAP
consistently for long durations, others use it much less, reflecting
differences in adherence or treatment habits.

\begin{Shaded}
\begin{Highlighting}[]
\DocumentationTok{\#\#5B Proportion of Participants with Average Daily CPAP Use Less Than 3 Hours}

\CommentTok{\# Calculate the proportion of participants with average daily CPAP use \textless{} 3 hours}
\NormalTok{proportion\_less\_than\_3 }\OtherTok{\textless{}{-}} \FunctionTok{mean}\NormalTok{(Data}\SpecialCharTok{$}\StringTok{\textasciigrave{}}\AttributeTok{Average daily CPAP (hr/night)}\StringTok{\textasciigrave{}} \SpecialCharTok{\textless{}} \DecValTok{3}\NormalTok{, }
                               \AttributeTok{na.rm =} \ConstantTok{TRUE}\NormalTok{)}

\CommentTok{\# Report the proportion}
\NormalTok{proportion\_less\_than\_3}
\end{Highlighting}
\end{Shaded}

\begin{verbatim}
## [1] 0.2068966
\end{verbatim}

Interpretation: About 20.7\% of the participants in the study use their
CPAP machine for less than three hours per night. This suggests that a
significant portion of participants may not be adhering to the
recommended CPAP usage, which could potentially reduce the effectiveness
of the therapy in managing sleep apnea.

In simpler terms, roughly one in five people in this study are not using
their CPAP machine long enough each night, which may make the treatment
less effective in improving their sleep and breathing.

\subsection{6A.Compute and Report the Mean and Standard Deviation for
BMI}\label{a.compute-and-report-the-mean-and-standard-deviation-for-bmi}

\begin{Shaded}
\begin{Highlighting}[]
\CommentTok{\# Calculate the mean and standard deviation for BMI}
\NormalTok{mean\_bmi }\OtherTok{\textless{}{-}} \FunctionTok{mean}\NormalTok{(Data}\SpecialCharTok{$}\NormalTok{BMI, }\AttributeTok{na.rm =} \ConstantTok{TRUE}\NormalTok{)}
\NormalTok{sd\_bmi }\OtherTok{\textless{}{-}} \FunctionTok{sd}\NormalTok{(Data}\SpecialCharTok{$}\NormalTok{BMI, }\AttributeTok{na.rm =} \ConstantTok{TRUE}\NormalTok{)}

\CommentTok{\# Report the results}
\NormalTok{mean\_bmi}
\end{Highlighting}
\end{Shaded}

\begin{verbatim}
## [1] 42.18415
\end{verbatim}

\begin{Shaded}
\begin{Highlighting}[]
\NormalTok{sd\_bmi}
\end{Highlighting}
\end{Shaded}

\begin{verbatim}
## [1] 7.206295
\end{verbatim}

\subsection{6B. Estimate and Report the Proportion of Participants Who
Are Considered Obese (BMI ≥
30)}\label{b.-estimate-and-report-the-proportion-of-participants-who-are-considered-obese-bmi-30}

\begin{Shaded}
\begin{Highlighting}[]
\CommentTok{\# Calculate the proportion of participants who are considered obese (BMI \textgreater{}= 30)}
\NormalTok{proportion\_obese }\OtherTok{\textless{}{-}} \FunctionTok{mean}\NormalTok{(Data}\SpecialCharTok{$}\NormalTok{BMI }\SpecialCharTok{\textgreater{}=} \DecValTok{30}\NormalTok{, }\AttributeTok{na.rm =} \ConstantTok{TRUE}\NormalTok{)}

\CommentTok{\# Report the proportion}
\NormalTok{proportion\_obese}
\end{Highlighting}
\end{Shaded}

\begin{verbatim}
## [1] 0.9597701
\end{verbatim}

Interpretation: The individuals in this study have an average BMI of
42.18, which is far higher than the normal healthy range of 18.5 to
24.9. The standard deviation of 7.21 indicates considerable variability
in BMI among participants.

Furthermore, about 96\% of participants had a BMI of 30 or higher,
classifying them as obese. This finding suggests that the majority of
individuals in this study are overweight or obese, an important factor
to consider when evaluating potential health outcomes and risks
associated with obesity.

\subsection{7A.Among those who are adherent to CPAP, describe the
distribution of the variable AHI with appropriate statistics and data
visualizations. Then, summarize your findings in 2-3 sentences in plain
language suitable for a non-statistical
audience.}\label{a.among-those-who-are-adherent-to-cpap-describe-the-distribution-of-the-variable-ahi-with-appropriate-statistics-and-data-visualizations.-then-summarize-your-findings-in-2-3-sentences-in-plain-language-suitable-for-a-non-statistical-audience.}

\begin{Shaded}
\begin{Highlighting}[]
\DocumentationTok{\#\#7A}

\CommentTok{\# Filter for adherent participants}
\NormalTok{adherent\_data }\OtherTok{\textless{}{-}}\NormalTok{ Data }\SpecialCharTok{\%\textgreater{}\%}
  \FunctionTok{filter}\NormalTok{(adherence }\SpecialCharTok{==} \StringTok{"Adherent"}\NormalTok{)}

\CommentTok{\# Summary statistics for AHI}
\FunctionTok{summary}\NormalTok{(adherent\_data}\SpecialCharTok{$}\NormalTok{AHI)}
\end{Highlighting}
\end{Shaded}

\begin{verbatim}
##    Min. 1st Qu.  Median    Mean 3rd Qu.    Max. 
##   15.00   19.50   26.75   34.49   43.65  112.70
\end{verbatim}

\begin{Shaded}
\begin{Highlighting}[]
\NormalTok{mean\_ahi }\OtherTok{\textless{}{-}} \FunctionTok{mean}\NormalTok{(adherent\_data}\SpecialCharTok{$}\NormalTok{AHI, }\AttributeTok{na.rm =} \ConstantTok{TRUE}\NormalTok{)}
\NormalTok{sd\_ahi }\OtherTok{\textless{}{-}} \FunctionTok{sd}\NormalTok{(adherent\_data}\SpecialCharTok{$}\NormalTok{AHI, }\AttributeTok{na.rm =} \ConstantTok{TRUE}\NormalTok{)}

\FunctionTok{cat}\NormalTok{(}\StringTok{"Mean AHI:"}\NormalTok{, mean\_ahi, }\StringTok{"}\SpecialCharTok{\textbackslash{}n}\StringTok{"}\NormalTok{)}
\end{Highlighting}
\end{Shaded}

\begin{verbatim}
## Mean AHI: 34.48664
\end{verbatim}

\begin{Shaded}
\begin{Highlighting}[]
\FunctionTok{cat}\NormalTok{(}\StringTok{"SD AHI:"}\NormalTok{, sd\_ahi, }\StringTok{"}\SpecialCharTok{\textbackslash{}n}\StringTok{"}\NormalTok{)}
\end{Highlighting}
\end{Shaded}

\begin{verbatim}
## SD AHI: 21.20208
\end{verbatim}

\begin{Shaded}
\begin{Highlighting}[]
\CommentTok{\# Histogram}
\FunctionTok{hist}\NormalTok{(adherent\_data}\SpecialCharTok{$}\NormalTok{AHI,}
     \AttributeTok{main =} \StringTok{"Distribution of AHI (Adherent Participants)"}\NormalTok{,}
     \AttributeTok{xlab =} \StringTok{"Apnea{-}Hypopnea Index (AHI)"}\NormalTok{,}
     \AttributeTok{col =} \StringTok{"skyblue"}\NormalTok{, }\AttributeTok{border =} \StringTok{"white"}\NormalTok{)}
\end{Highlighting}
\end{Shaded}

\pandocbounded{\includegraphics[keepaspectratio]{Activity-2_trial1_files/figure-latex/unnamed-chunk-9-1.pdf}}

Interpretation: For the adherent CPAP participants, the Apnea--Hypopnea
Index (AHI) has an average value of approximately 34.5, with a standard
deviation of 21.2. The AHI values range from 15 to 112.7, indicating a
wide spread. The histogram shows a right-skewed distribution, suggesting
that a few participants with very high AHI values substantially
influence the overall data pattern.

In plain English: Among people who regularly use CPAP, the AHI usually
averages around 34.5, but it can vary widely---from 15 to over 110. This
means that while many participants have moderate AHI scores, some
experience much more severe sleep apnea. Because a small group has very
high AHI values, the data are unevenly distributed, making the average
appear higher.

\subsection{7B. Among those who are adherent to CPAP, describe the
sampling distribution based on samples of size 30 from the variable AHI
based on theory with appropriate statistics and data visualizations.
Then, summarize findings in 2-3 sentences in plain language suitable for
a non-statistical
audience.}\label{b.-among-those-who-are-adherent-to-cpap-describe-the-sampling-distribution-based-on-samples-of-size-30-from-the-variable-ahi-based-on-theory-with-appropriate-statistics-and-data-visualizations.-then-summarize-findings-in-2-3-sentences-in-plain-language-suitable-for-a-non-statistical-audience.}

\begin{Shaded}
\begin{Highlighting}[]
\DocumentationTok{\#\#7B}

\FunctionTok{library}\NormalTok{(dplyr)}
\FunctionTok{library}\NormalTok{(ggplot2)}
\CommentTok{\# 1) Subset + ensure numeric}
\NormalTok{adherent\_data }\OtherTok{\textless{}{-}}\NormalTok{ Data }\SpecialCharTok{\%\textgreater{}\%}
  \FunctionTok{filter}\NormalTok{(adherence }\SpecialCharTok{==} \StringTok{"Adherent"}\NormalTok{) }\SpecialCharTok{\%\textgreater{}\%}
  \FunctionTok{mutate}\NormalTok{(}\AttributeTok{AHI =} \FunctionTok{as.numeric}\NormalTok{(AHI))}

\CommentTok{\# 2) Population (group) parameters for AHI among adherent participants}
\NormalTok{mu\_hat  }\OtherTok{\textless{}{-}} \FunctionTok{mean}\NormalTok{(adherent\_data}\SpecialCharTok{$}\NormalTok{AHI, }\AttributeTok{na.rm =} \ConstantTok{TRUE}\NormalTok{)   }
\NormalTok{sd\_hat  }\OtherTok{\textless{}{-}} \FunctionTok{sd}\NormalTok{(adherent\_data}\SpecialCharTok{$}\NormalTok{AHI,   }\AttributeTok{na.rm =} \ConstantTok{TRUE}\NormalTok{)  }
\NormalTok{n\_theory }\OtherTok{\textless{}{-}} \DecValTok{30}
\NormalTok{SE\_theory }\OtherTok{\textless{}{-}}\NormalTok{ sd\_hat }\SpecialCharTok{/} \FunctionTok{sqrt}\NormalTok{(n\_theory)}

\FunctionTok{cat}\NormalTok{(}
  \StringTok{"Adherent AHI (theory, n = 30)}\SpecialCharTok{\textbackslash{}n}\StringTok{"}\NormalTok{,}
  \StringTok{"Mean (μ̂):"}\NormalTok{, }\FunctionTok{round}\NormalTok{(mu\_hat, }\DecValTok{2}\NormalTok{), }\StringTok{"}\SpecialCharTok{\textbackslash{}n}\StringTok{"}\NormalTok{,}
  \StringTok{"SD (σ̂):"}\NormalTok{, }\FunctionTok{round}\NormalTok{(sd\_hat, }\DecValTok{2}\NormalTok{), }\StringTok{"}\SpecialCharTok{\textbackslash{}n}\StringTok{"}\NormalTok{,}
  \StringTok{"Standard Error (SE = σ̂/sqrt(30)):"}\NormalTok{, }\FunctionTok{round}\NormalTok{(SE\_theory, }\DecValTok{3}\NormalTok{), }\StringTok{"}\SpecialCharTok{\textbackslash{}n}\StringTok{"}
\NormalTok{)}
\end{Highlighting}
\end{Shaded}

\begin{verbatim}
## Adherent AHI (theory, n = 30)
##  Mean (μ̂): 34.49 
##  SD (σ̂): 21.2 
##  Standard Error (SE = σ̂/sqrt(30)): 3.871
\end{verbatim}

\begin{Shaded}
\begin{Highlighting}[]
\CommentTok{\# 3) Visual 1 (theoretical normal curve for X̄, no data histogram)}
\NormalTok{xgrid }\OtherTok{\textless{}{-}} \FunctionTok{data.frame}\NormalTok{(}
  \AttributeTok{x =} \FunctionTok{seq}\NormalTok{(mu\_hat }\SpecialCharTok{{-}} \DecValTok{4}\SpecialCharTok{*}\NormalTok{SE\_theory, mu\_hat }\SpecialCharTok{+} \DecValTok{4}\SpecialCharTok{*}\NormalTok{SE\_theory, }\AttributeTok{length.out =} \DecValTok{400}\NormalTok{)}
\NormalTok{)}
\FunctionTok{ggplot}\NormalTok{(xgrid, }\FunctionTok{aes}\NormalTok{(x)) }\SpecialCharTok{+}
  \FunctionTok{stat\_function}\NormalTok{(}\AttributeTok{fun =}\NormalTok{ dnorm, }\AttributeTok{args =} \FunctionTok{list}\NormalTok{(}\AttributeTok{mean =}\NormalTok{ mu\_hat, }\AttributeTok{sd =}\NormalTok{ SE\_theory)) }\SpecialCharTok{+}
  \FunctionTok{labs}\NormalTok{(}
    \AttributeTok{title =} \StringTok{"Theoretical Sampling Distribution of Mean AHI (Adherent, n = 30)"}\NormalTok{,}
    \AttributeTok{x =} \StringTok{"Sample Mean AHI (X̄)"}\NormalTok{,}
    \AttributeTok{y =} \StringTok{"Density"}
\NormalTok{  ) }\SpecialCharTok{+}
  \FunctionTok{theme\_minimal}\NormalTok{()}
\end{Highlighting}
\end{Shaded}

\pandocbounded{\includegraphics[keepaspectratio]{Activity-2_trial1_files/figure-latex/unnamed-chunk-10-1.pdf}}

\begin{Shaded}
\begin{Highlighting}[]
\CommentTok{\#Visual 2: simulate to illustrate CLT and overlay theory}
\FunctionTok{set.seed}\NormalTok{(}\DecValTok{2025}\NormalTok{)}
\NormalTok{B }\OtherTok{\textless{}{-}} \DecValTok{1000}
\NormalTok{samp\_means }\OtherTok{\textless{}{-}} \FunctionTok{replicate}\NormalTok{(}
\NormalTok{  B,}
  \FunctionTok{mean}\NormalTok{(}\FunctionTok{sample}\NormalTok{(adherent\_data}\SpecialCharTok{$}\NormalTok{AHI, }\AttributeTok{size =}\NormalTok{ n\_theory, }\AttributeTok{replace =} \ConstantTok{TRUE}\NormalTok{), }\AttributeTok{na.rm =} \ConstantTok{TRUE}\NormalTok{)}
\NormalTok{)}

\FunctionTok{ggplot}\NormalTok{(}\FunctionTok{data.frame}\NormalTok{(samp\_means), }\FunctionTok{aes}\NormalTok{(}\AttributeTok{x =}\NormalTok{ samp\_means)) }\SpecialCharTok{+}
  \FunctionTok{geom\_histogram}\NormalTok{(}\FunctionTok{aes}\NormalTok{(}\AttributeTok{y =} \FunctionTok{after\_stat}\NormalTok{(density)), }\AttributeTok{bins =} \DecValTok{30}\NormalTok{, }\AttributeTok{fill =} \StringTok{"steelblue"}\NormalTok{, }
                 \AttributeTok{color =} \StringTok{"white"}\NormalTok{, }\AttributeTok{alpha =} \FloatTok{0.75}\NormalTok{) }\SpecialCharTok{+}
  \FunctionTok{stat\_function}\NormalTok{(}\AttributeTok{fun =}\NormalTok{ dnorm, }\AttributeTok{args =} \FunctionTok{list}\NormalTok{(}\AttributeTok{mean =}\NormalTok{ mu\_hat, }\AttributeTok{sd =}\NormalTok{ SE\_theory),}
                \AttributeTok{linewidth =} \DecValTok{1}\NormalTok{, }\AttributeTok{color =} \StringTok{"firebrick"}\NormalTok{) }\SpecialCharTok{+}
  \FunctionTok{labs}\NormalTok{(}
    \AttributeTok{title =} \StringTok{"Sampling Distribution of Mean AHI (Adherent, n = 30)}\SpecialCharTok{\textbackslash{}n}\StringTok{Histogram of }
\StringTok{    1,000 sample means with theoretical normal overlay"}\NormalTok{,}
    \AttributeTok{x =} \StringTok{"Sample Mean AHI (X̄)"}\NormalTok{,}
    \AttributeTok{y =} \StringTok{"Density"}
\NormalTok{  ) }\SpecialCharTok{+}
  \FunctionTok{theme\_minimal}\NormalTok{()}
\end{Highlighting}
\end{Shaded}

\pandocbounded{\includegraphics[keepaspectratio]{Activity-2_trial1_files/figure-latex/unnamed-chunk-10-2.pdf}}

Interpretation: If we repeatedly take random groups of 30 adherent CPAP
users and calculate their average AHI, those averages would tend to
cluster around 34.5. The standard error of 3.9 indicates that the sample
means would typically vary by about 3.9 AHI points from one sample to
another. In other words, most of the group averages would lie close to
the true mean, forming a bell-shaped (normal) distribution.

In short, when drawing samples of 30 adherent users, the average AHI for
each group would generally be close to 34.5, with only moderate
variation between samples.

\subsection{7C. Randomly draw 1,000 samples of size 30 (with
replacement) from the 128 AHI scores of adherent participants. For each
sample, compute the sample mean AHI. Use these 1,000 sample means
to:}\label{c.-randomly-draw-1000-samples-of-size-30-with-replacement-from-the-128-ahi-scores-of-adherent-participants.-for-each-sample-compute-the-sample-mean-ahi.-use-these-1000-sample-means-to}

\begin{Shaded}
\begin{Highlighting}[]
\CommentTok{\#7C}
\CommentTok{\# Ensure the AHI column is numeric}
\NormalTok{adherent\_data}\SpecialCharTok{$}\NormalTok{AHI }\OtherTok{\textless{}{-}} \FunctionTok{as.numeric}\NormalTok{(adherent\_data}\SpecialCharTok{$}\NormalTok{AHI)}

\CommentTok{\# Define the parameters}
\NormalTok{MY\_DATA\_1 }\OtherTok{\textless{}{-}}\NormalTok{ adherent\_data}
\NormalTok{VARIABLE }\OtherTok{\textless{}{-}} \StringTok{"AHI"}          \CommentTok{\# The variable to sample}
\NormalTok{SAMPLES }\OtherTok{\textless{}{-}} \DecValTok{1000}            \CommentTok{\# Number of samples}
\NormalTok{SIZE }\OtherTok{\textless{}{-}} \DecValTok{30}                 \CommentTok{\# Sample size}

\CommentTok{\# Initialize an empty vector to store the means}
\NormalTok{meanValues }\OtherTok{\textless{}{-}} \FunctionTok{numeric}\NormalTok{(SAMPLES)}

\CommentTok{\# Sampling loop (1000 samples, sample size 30)}
\ControlFlowTok{for}\NormalTok{ (i }\ControlFlowTok{in} \DecValTok{1}\SpecialCharTok{:}\NormalTok{SAMPLES) \{}
\NormalTok{  sampSpots }\OtherTok{\textless{}{-}} \FunctionTok{sample}\NormalTok{(}\DecValTok{1}\SpecialCharTok{:}\FunctionTok{nrow}\NormalTok{(MY\_DATA\_1), }\AttributeTok{size =}\NormalTok{ SIZE, }\AttributeTok{replace =} \ConstantTok{TRUE}\NormalTok{)}
\NormalTok{  thisSamp }\OtherTok{\textless{}{-}}\NormalTok{ MY\_DATA\_1[sampSpots, VARIABLE]}
  
  \CommentTok{\# Ensure the sample is numeric and handle any NA values}
\NormalTok{  thisSamp }\OtherTok{\textless{}{-}}\NormalTok{ thisSamp[}\SpecialCharTok{!}\FunctionTok{is.na}\NormalTok{(thisSamp)]}
  
  \ControlFlowTok{if}\NormalTok{(}\FunctionTok{length}\NormalTok{(thisSamp) }\SpecialCharTok{==}\NormalTok{ SIZE) \{  }\CommentTok{\# Check if the sample is the correct size}
\NormalTok{    meanValues[i] }\OtherTok{\textless{}{-}} \FunctionTok{mean}\NormalTok{(thisSamp)}
\NormalTok{  \} }\ControlFlowTok{else}\NormalTok{ \{}
\NormalTok{    meanValues[i] }\OtherTok{\textless{}{-}} \ConstantTok{NA}  \CommentTok{\# If the sample is not valid, mark as NA}
\NormalTok{  \}}
\NormalTok{\}}

\CommentTok{\# Check if valid sample means are calculated}
\ControlFlowTok{if}\NormalTok{(}\FunctionTok{sum}\NormalTok{(}\SpecialCharTok{!}\FunctionTok{is.na}\NormalTok{(meanValues)) }\SpecialCharTok{==} \DecValTok{0}\NormalTok{) \{}
  \FunctionTok{stop}\NormalTok{(}\StringTok{"No valid means were calculated. Check the sampling process."}\NormalTok{)}
\NormalTok{\}}

\CommentTok{\# Calculate and report mean and SD of the sampling distribution}
\NormalTok{mean\_sampling\_1 }\OtherTok{\textless{}{-}} \FunctionTok{mean}\NormalTok{(meanValues, }\AttributeTok{na.rm =} \ConstantTok{TRUE}\NormalTok{)}
\NormalTok{sd\_sampling\_1 }\OtherTok{\textless{}{-}} \FunctionTok{sd}\NormalTok{(meanValues, }\AttributeTok{na.rm =} \ConstantTok{TRUE}\NormalTok{)}

\FunctionTok{cat}\NormalTok{(}\StringTok{"Mean of sample means:"}\NormalTok{, mean\_sampling\_1, }\StringTok{"}\SpecialCharTok{\textbackslash{}n}\StringTok{"}\NormalTok{)}
\end{Highlighting}
\end{Shaded}

\begin{verbatim}
## Mean of sample means: 34.49511
\end{verbatim}

\begin{Shaded}
\begin{Highlighting}[]
\FunctionTok{cat}\NormalTok{(}\StringTok{"Standard deviation of sample means (Standard Error):"}\NormalTok{, sd\_sampling\_1, }\StringTok{"}\SpecialCharTok{\textbackslash{}n}\StringTok{"}\NormalTok{)}
\end{Highlighting}
\end{Shaded}

\begin{verbatim}
## Standard deviation of sample means (Standard Error): 3.798762
\end{verbatim}

\begin{Shaded}
\begin{Highlighting}[]
\CommentTok{\# Plot the histogram and overlay normal curve}
\FunctionTok{library}\NormalTok{(ggplot2)}
\FunctionTok{ggplot}\NormalTok{(}\FunctionTok{data.frame}\NormalTok{(meanValues), }\FunctionTok{aes}\NormalTok{(}\AttributeTok{x =}\NormalTok{ meanValues)) }\SpecialCharTok{+}
  \FunctionTok{geom\_histogram}\NormalTok{(}\FunctionTok{aes}\NormalTok{(}\AttributeTok{y =} \FunctionTok{after\_stat}\NormalTok{(density)), }\AttributeTok{bins =} \DecValTok{30}\NormalTok{, }\AttributeTok{fill =} \StringTok{"skyblue"}\NormalTok{, }
                 \AttributeTok{color =} \StringTok{"white"}\NormalTok{) }\SpecialCharTok{+}
  \FunctionTok{stat\_function}\NormalTok{(}\AttributeTok{fun =}\NormalTok{ dnorm, }\AttributeTok{args =} \FunctionTok{list}\NormalTok{(}\AttributeTok{mean =}\NormalTok{ mean\_sampling\_1, }
                                         \AttributeTok{sd =}\NormalTok{ sd\_sampling\_1), }
                \AttributeTok{color =} \StringTok{"darkred"}\NormalTok{, }\AttributeTok{linewidth =} \DecValTok{1}\NormalTok{) }\SpecialCharTok{+}
  \FunctionTok{labs}\NormalTok{(}\AttributeTok{title =} \StringTok{"Sampling Distribution of Mean AHI (n = 30, 1000 samples)"}\NormalTok{, }
       \AttributeTok{x =} \StringTok{"Sample Mean AHI"}\NormalTok{, }\AttributeTok{y =} \StringTok{"Density"}\NormalTok{) }\SpecialCharTok{+}
  \FunctionTok{theme\_minimal}\NormalTok{()}
\end{Highlighting}
\end{Shaded}

\pandocbounded{\includegraphics[keepaspectratio]{Activity-2_trial1_files/figure-latex/unnamed-chunk-11-1.pdf}}

Interpretation: The histogram of the 1,000 sample means forms a smooth,
symmetric, bell-shaped curve centered around 34.5, with the red normal
curve closely overlapping it. This shows that the sample means follow an
approximately normal distribution with only moderate variability,
meaning that averages from groups of 30 adherent participants are highly
consistent.

The observed sampling distribution matches the theoretical distribution
predicted earlier, with nearly the same mean (≈ 34.5) and standard error
(≈ 3.9). This agreement provides strong evidence of the Central Limit
Theorem in action---even though individual AHI scores vary widely, the
averages from groups of 30 adherent users remain stable and normally
distributed around the true mean.

\subsection{7D. Among those who are non-adherent to CPAP, describe the
sampling distribution based on samples of size 100 from the variable AHI
based on theory with appropriate statistics and data visualizations.
Then summarize your findings in 2-3 sentences in plain language suitable
for a non-statistical
audience.}\label{d.-among-those-who-are-non-adherent-to-cpap-describe-the-sampling-distribution-based-on-samples-of-size-100-from-the-variable-ahi-based-on-theory-with-appropriate-statistics-and-data-visualizations.-then-summarize-your-findings-in-2-3-sentences-in-plain-language-suitable-for-a-non-statistical-audience.}

\begin{Shaded}
\begin{Highlighting}[]
\DocumentationTok{\#\# 7D}
\CommentTok{\# Load necessary libraries}
\FunctionTok{library}\NormalTok{(dplyr)}
\FunctionTok{library}\NormalTok{(ggplot2)}

\CommentTok{\# Filter for non{-}adherent participants}
\NormalTok{non\_adherent\_data }\OtherTok{\textless{}{-}}\NormalTok{ Data }\SpecialCharTok{\%\textgreater{}\%}
  \FunctionTok{filter}\NormalTok{(adherence }\SpecialCharTok{==} \StringTok{"Non{-}adherent"}\NormalTok{)}

\CommentTok{\# Ensure the AHI column is numeric}
\NormalTok{non\_adherent\_data}\SpecialCharTok{$}\NormalTok{AHI }\OtherTok{\textless{}{-}} \FunctionTok{as.numeric}\NormalTok{(non\_adherent\_data}\SpecialCharTok{$}\NormalTok{AHI)}

\CommentTok{\# Calculate population parameters (mean and standard deviation of AHI)}
\NormalTok{population\_mean }\OtherTok{\textless{}{-}} \FunctionTok{mean}\NormalTok{(non\_adherent\_data}\SpecialCharTok{$}\NormalTok{AHI, }\AttributeTok{na.rm =} \ConstantTok{TRUE}\NormalTok{)}
\NormalTok{population\_sd }\OtherTok{\textless{}{-}} \FunctionTok{sd}\NormalTok{(non\_adherent\_data}\SpecialCharTok{$}\NormalTok{AHI, }\AttributeTok{na.rm =} \ConstantTok{TRUE}\NormalTok{)}

\CommentTok{\# Parameters for sampling}
\NormalTok{sample\_size }\OtherTok{\textless{}{-}} \DecValTok{100}  \CommentTok{\# Sample size}
\NormalTok{n\_samples }\OtherTok{\textless{}{-}} \DecValTok{1000}   \CommentTok{\# Number of samples}

\CommentTok{\# Calculate the standard error}
\NormalTok{standard\_error }\OtherTok{\textless{}{-}}\NormalTok{ population\_sd }\SpecialCharTok{/} \FunctionTok{sqrt}\NormalTok{(sample\_size)}

\CommentTok{\# Simulate the sampling distribution of the mean}
\FunctionTok{set.seed}\NormalTok{(}\DecValTok{123}\NormalTok{)  }\CommentTok{\# Set a seed for reproducibility}
\NormalTok{sample\_means }\OtherTok{\textless{}{-}} \FunctionTok{numeric}\NormalTok{(n\_samples)}

\ControlFlowTok{for}\NormalTok{ (i }\ControlFlowTok{in} \DecValTok{1}\SpecialCharTok{:}\NormalTok{n\_samples) \{}
  \CommentTok{\# Randomly sample 100 values from the population with replacement}
\NormalTok{  sample\_data }\OtherTok{\textless{}{-}} \FunctionTok{sample}\NormalTok{(non\_adherent\_data}\SpecialCharTok{$}\NormalTok{AHI, }\AttributeTok{size =}\NormalTok{ sample\_size, }\AttributeTok{replace =} \ConstantTok{TRUE}\NormalTok{)}
\NormalTok{  sample\_means[i] }\OtherTok{\textless{}{-}} \FunctionTok{mean}\NormalTok{(sample\_data, }\AttributeTok{na.rm =} \ConstantTok{TRUE}\NormalTok{)}
\NormalTok{\}}

\CommentTok{\# Calculate mean and standard deviation of the sampling distribution}
\NormalTok{mean\_sampling }\OtherTok{\textless{}{-}} \FunctionTok{mean}\NormalTok{(sample\_means, }\AttributeTok{na.rm =} \ConstantTok{TRUE}\NormalTok{)}
\NormalTok{sd\_sampling }\OtherTok{\textless{}{-}} \FunctionTok{sd}\NormalTok{(sample\_means, }\AttributeTok{na.rm =} \ConstantTok{TRUE}\NormalTok{)}

\FunctionTok{cat}\NormalTok{(}\StringTok{"Theoretical Mean of the Sampling Distribution:"}\NormalTok{, mean\_sampling, }\StringTok{"}\SpecialCharTok{\textbackslash{}n}\StringTok{"}\NormalTok{)}
\end{Highlighting}
\end{Shaded}

\begin{verbatim}
## Theoretical Mean of the Sampling Distribution: 35.49504
\end{verbatim}

\begin{Shaded}
\begin{Highlighting}[]
\FunctionTok{cat}\NormalTok{(}\StringTok{"Theoretical Standard Deviation of the Sampling Distribution }
\StringTok{    (Standard Error):"}\NormalTok{, sd\_sampling, }\StringTok{"}\SpecialCharTok{\textbackslash{}n}\StringTok{"}\NormalTok{)}
\end{Highlighting}
\end{Shaded}

\begin{verbatim}
## Theoretical Standard Deviation of the Sampling Distribution 
##     (Standard Error): 1.881024
\end{verbatim}

\begin{Shaded}
\begin{Highlighting}[]
\CommentTok{\# Plot the histogram of the sample means and overlay a normal distribution curve}
\FunctionTok{ggplot}\NormalTok{(}\FunctionTok{data.frame}\NormalTok{(sample\_means), }\FunctionTok{aes}\NormalTok{(}\AttributeTok{x =}\NormalTok{ sample\_means)) }\SpecialCharTok{+}
  \FunctionTok{geom\_histogram}\NormalTok{(}\FunctionTok{aes}\NormalTok{(}\AttributeTok{y =} \FunctionTok{after\_stat}\NormalTok{(density)), }\AttributeTok{bins =} \DecValTok{30}\NormalTok{, }\AttributeTok{fill =} \StringTok{"skyblue"}\NormalTok{, }
                 \AttributeTok{color =} \StringTok{"white"}\NormalTok{, }\AttributeTok{alpha =} \FloatTok{0.7}\NormalTok{) }\SpecialCharTok{+}
  \FunctionTok{stat\_function}\NormalTok{(}\AttributeTok{fun =}\NormalTok{ dnorm, }
                \AttributeTok{args =} \FunctionTok{list}\NormalTok{(}\AttributeTok{mean =}\NormalTok{ population\_mean, }\AttributeTok{sd =}\NormalTok{ standard\_error), }
                \AttributeTok{color =} \StringTok{"darkred"}\NormalTok{, }\AttributeTok{linewidth =} \DecValTok{1}\NormalTok{) }\SpecialCharTok{+}  \CommentTok{\# Overlay normal curve}
  \FunctionTok{labs}\NormalTok{(}\AttributeTok{title =} \StringTok{"Sampling Distribution of Mean AHI (n = 100, 1000 samples)"}\NormalTok{, }
       \AttributeTok{x =} \StringTok{"Sample Mean AHI"}\NormalTok{, }\AttributeTok{y =} \StringTok{"Density"}\NormalTok{) }\SpecialCharTok{+}
  \FunctionTok{theme\_minimal}\NormalTok{()}
\end{Highlighting}
\end{Shaded}

\pandocbounded{\includegraphics[keepaspectratio]{Activity-2_trial1_files/figure-latex/unnamed-chunk-12-1.pdf}}

Interpretation: If we repeatedly take random groups of 100 non-adherent
CPAP users and calculate their average AHI, those averages would cluster
around 35.6. The standard error of about 2 AHI points indicates that the
sample means would vary only slightly from one sample to another. This
means most group averages would fall close to the true mean, forming a
bell-shaped (normal) distribution.

In short, when taking samples of 100 non-adherent users, the average AHI
is estimated very precisely around 35--36, showing that larger sample
sizes produce more stable and reliable estimates of the true mean.

\subsection{7E. Randomly draw 1,000 samples of size 100 (with
replacement) from the 46 AHI scores of non-adherent participants. For
each sample, compute the sample mean
AHI.}\label{e.-randomly-draw-1000-samples-of-size-100-with-replacement-from-the-46-ahi-scores-of-non-adherent-participants.-for-each-sample-compute-the-sample-mean-ahi.}

\begin{Shaded}
\begin{Highlighting}[]
\DocumentationTok{\#\#7E}
\CommentTok{\# Replace the following with actual values:}
\NormalTok{MY\_DATA\_2 }\OtherTok{\textless{}{-}}\NormalTok{ non\_adherent\_data  }
\NormalTok{VARIABLE }\OtherTok{\textless{}{-}} \StringTok{"AHI"}            
\NormalTok{SAMPLES }\OtherTok{\textless{}{-}} \DecValTok{1000}              
\NormalTok{SIZE }\OtherTok{\textless{}{-}} \DecValTok{100}                  

\CommentTok{\# Ensure the AHI column is numeric}
\NormalTok{non\_adherent\_data}\SpecialCharTok{$}\NormalTok{AHI }\OtherTok{\textless{}{-}} \FunctionTok{as.numeric}\NormalTok{(non\_adherent\_data}\SpecialCharTok{$}\NormalTok{AHI)}
\CommentTok{\# Ensure the variable is numeric }
\NormalTok{MY\_DATA\_2[[VARIABLE]] }\OtherTok{\textless{}{-}} \FunctionTok{as.numeric}\NormalTok{(MY\_DATA\_2[[VARIABLE]])}
\CommentTok{\# Initialize an empty vector to store sample means}
\NormalTok{meanValues }\OtherTok{\textless{}{-}} \ConstantTok{NULL}

\CommentTok{\# Loop to draw SAMPLES number of samples and calculate mean AHI for each sample}
\ControlFlowTok{for}\NormalTok{ (i }\ControlFlowTok{in} \DecValTok{1}\SpecialCharTok{:}\NormalTok{SAMPLES) \{}
  \CommentTok{\# Sample with replacement}
\NormalTok{  sampSpots }\OtherTok{\textless{}{-}} \FunctionTok{sample}\NormalTok{(}\AttributeTok{x =} \DecValTok{1}\SpecialCharTok{:}\FunctionTok{nrow}\NormalTok{(MY\_DATA\_2), }\AttributeTok{size =}\NormalTok{ SIZE, }\AttributeTok{replace =} \ConstantTok{TRUE}\NormalTok{)}
  
  \CommentTok{\# Extract the sample data for the variable specified}
\NormalTok{  thisSamp }\OtherTok{\textless{}{-}}\NormalTok{ MY\_DATA\_2[sampSpots, ][[VARIABLE]] }
  

  
\CommentTok{\# Calculate and store the mean of the sample}
\NormalTok{  meanValues }\OtherTok{\textless{}{-}} \FunctionTok{c}\NormalTok{(meanValues, }\FunctionTok{mean}\NormalTok{(thisSamp, }\AttributeTok{na.rm =} \ConstantTok{TRUE}\NormalTok{))  }
\NormalTok{\}}


\CommentTok{\# Calculate and print the mean and sd of the sampling distribution}
\NormalTok{mean\_sampling\_2 }\OtherTok{\textless{}{-}} \FunctionTok{mean}\NormalTok{(meanValues, }\AttributeTok{na.rm =} \ConstantTok{TRUE}\NormalTok{)}
\NormalTok{sd\_sampling\_2 }\OtherTok{\textless{}{-}} \FunctionTok{sd}\NormalTok{(meanValues, }\AttributeTok{na.rm =} \ConstantTok{TRUE}\NormalTok{)}

\FunctionTok{cat}\NormalTok{(}\StringTok{"Mean of the sampling distribution:"}\NormalTok{, mean\_sampling\_2, }\StringTok{"}\SpecialCharTok{\textbackslash{}n}\StringTok{"}\NormalTok{)}
\end{Highlighting}
\end{Shaded}

\begin{verbatim}
## Mean of the sampling distribution: 35.61692
\end{verbatim}

\begin{Shaded}
\begin{Highlighting}[]
\FunctionTok{cat}\NormalTok{(}\StringTok{"Standard deviation of the sampling distribution (Standard Error):"}\NormalTok{, sd\_sampling\_2, }\StringTok{"}\SpecialCharTok{\textbackslash{}n}\StringTok{"}\NormalTok{)}
\end{Highlighting}
\end{Shaded}

\begin{verbatim}
## Standard deviation of the sampling distribution (Standard Error): 2.021068
\end{verbatim}

\begin{Shaded}
\begin{Highlighting}[]
\CommentTok{\# Plot the histogram of sample means and overlay a normal distribution curve}
\FunctionTok{library}\NormalTok{(ggplot2)}

\FunctionTok{ggplot}\NormalTok{(}\FunctionTok{data.frame}\NormalTok{(meanValues), }\FunctionTok{aes}\NormalTok{(}\AttributeTok{x =}\NormalTok{ meanValues)) }\SpecialCharTok{+}
  \FunctionTok{geom\_histogram}\NormalTok{(}\FunctionTok{aes}\NormalTok{(}\AttributeTok{y =} \FunctionTok{after\_stat}\NormalTok{(density)), }\AttributeTok{bins =} \DecValTok{30}\NormalTok{, }\AttributeTok{fill =} \StringTok{"skyblue"}\NormalTok{, }\AttributeTok{color =} \StringTok{"white"}\NormalTok{, }\AttributeTok{alpha =} \FloatTok{0.7}\NormalTok{) }\SpecialCharTok{+}
  \FunctionTok{stat\_function}\NormalTok{(}\AttributeTok{fun =}\NormalTok{ dnorm, }
                \AttributeTok{args =} \FunctionTok{list}\NormalTok{(}\AttributeTok{mean =}\NormalTok{ mean\_sampling\_2, }\AttributeTok{sd =}\NormalTok{ sd\_sampling\_2), }
                \AttributeTok{color =} \StringTok{"darkred"}\NormalTok{, }\AttributeTok{linewidth =} \DecValTok{1}\NormalTok{) }\SpecialCharTok{+}  \CommentTok{\# Overlay normal curve}
  \FunctionTok{labs}\NormalTok{(}\AttributeTok{title =} \StringTok{"Sampling Distribution of Mean AHI (n = 100, 1000 samples)"}\NormalTok{, }
       \AttributeTok{x =} \StringTok{"Sample Mean AHI"}\NormalTok{, }\AttributeTok{y =} \StringTok{"Density"}\NormalTok{) }\SpecialCharTok{+}
  \FunctionTok{theme\_minimal}\NormalTok{()}
\end{Highlighting}
\end{Shaded}

\pandocbounded{\includegraphics[keepaspectratio]{Activity-2_trial1_files/figure-latex/unnamed-chunk-13-1.pdf}}

Interpretation: The histogram of 1,000 sample averages for the
non-adherent group forms a smooth, bell-shaped curve centered around
35.6, with the red normal curve fitting closely on top. This indicates
that the sample means follow an approximately normal distribution. The
spread is small---about 2 AHI points---showing that the averages from
groups of 100 non-adherent participants vary very little between
samples.

Comparison with Part (D): This empirical sampling distribution,
generated from multiple simulations, closely matches the theoretical
distribution described in Part (D). Both have nearly identical means (≈
35.6) and standard errors (≈ 2), confirming that theoretical
expectations and simulated results agree.

In plain terms: Whether calculated mathematically or obtained through
resampling, the results show that the average AHI among non-adherent
users is stable, predictable, and normally distributed when the sample
size is large.

\subsection{7F. Compare the sampling distributions among adherent versus
non-adherent participants. How do the means and spreads of the two
sampling distributions compare? Why do you think they are the same or
different? Summarize your findings in 3-5 sentences in plain language
suitable for a non-statistical
audience.}\label{f.-compare-the-sampling-distributions-among-adherent-versus-non-adherent-participants.-how-do-the-means-and-spreads-of-the-two-sampling-distributions-compare-why-do-you-think-they-are-the-same-or-different-summarize-your-findings-in-3-5-sentences-in-plain-language-suitable-for-a-non-statistical-audience.}

\begin{Shaded}
\begin{Highlighting}[]
\DocumentationTok{\#\#7F}
\CommentTok{\#Filter for adherent and non{-}adherent participants}
\NormalTok{adherent\_data }\OtherTok{\textless{}{-}}\NormalTok{ Data }\SpecialCharTok{\%\textgreater{}\%} \FunctionTok{filter}\NormalTok{(adherence }\SpecialCharTok{==} \StringTok{"Adherent"}\NormalTok{)}
\NormalTok{non\_adherent\_data }\OtherTok{\textless{}{-}}\NormalTok{ Data }\SpecialCharTok{\%\textgreater{}\%} \FunctionTok{filter}\NormalTok{(adherence }\SpecialCharTok{==} \StringTok{"Non{-}adherent"}\NormalTok{)}

\CommentTok{\# Ensure AHI is numeric}
\NormalTok{adherent\_data}\SpecialCharTok{$}\NormalTok{AHI }\OtherTok{\textless{}{-}} \FunctionTok{as.numeric}\NormalTok{(adherent\_data}\SpecialCharTok{$}\NormalTok{AHI)}
\NormalTok{non\_adherent\_data}\SpecialCharTok{$}\NormalTok{AHI }\OtherTok{\textless{}{-}} \FunctionTok{as.numeric}\NormalTok{(non\_adherent\_data}\SpecialCharTok{$}\NormalTok{AHI)}

\CommentTok{\# Set parameters}
\NormalTok{SAMPLES }\OtherTok{\textless{}{-}} \DecValTok{1000}
\NormalTok{SIZE\_ADHERENT }\OtherTok{\textless{}{-}} \DecValTok{30}
\NormalTok{SIZE\_NONADHERENT }\OtherTok{\textless{}{-}} \DecValTok{100}

\CommentTok{\# Initialize vectors}
\NormalTok{sample\_means\_adherent }\OtherTok{\textless{}{-}} \FunctionTok{numeric}\NormalTok{(SAMPLES)}
\NormalTok{sample\_means\_non\_adherent }\OtherTok{\textless{}{-}} \FunctionTok{numeric}\NormalTok{(SAMPLES)}

\FunctionTok{set.seed}\NormalTok{(}\DecValTok{123}\NormalTok{)}

\CommentTok{\# Loop to calculate sample means}
\ControlFlowTok{for}\NormalTok{ (i }\ControlFlowTok{in} \DecValTok{1}\SpecialCharTok{:}\NormalTok{SAMPLES) \{}
\NormalTok{  sample\_adherent }\OtherTok{\textless{}{-}} \FunctionTok{sample}\NormalTok{(adherent\_data}\SpecialCharTok{$}\NormalTok{AHI, }\AttributeTok{size =}\NormalTok{ SIZE\_ADHERENT, }
                            \AttributeTok{replace =} \ConstantTok{TRUE}\NormalTok{)}
\NormalTok{  sample\_non\_adherent }\OtherTok{\textless{}{-}} \FunctionTok{sample}\NormalTok{(non\_adherent\_data}\SpecialCharTok{$}\NormalTok{AHI, }\AttributeTok{size =}\NormalTok{ SIZE\_NONADHERENT, }
                                \AttributeTok{replace =} \ConstantTok{TRUE}\NormalTok{)}
  
\NormalTok{  sample\_means\_adherent[i] }\OtherTok{\textless{}{-}} \FunctionTok{mean}\NormalTok{(sample\_adherent, }\AttributeTok{na.rm =} \ConstantTok{TRUE}\NormalTok{)}
\NormalTok{  sample\_means\_non\_adherent[i] }\OtherTok{\textless{}{-}} \FunctionTok{mean}\NormalTok{(sample\_non\_adherent, }\AttributeTok{na.rm =} \ConstantTok{TRUE}\NormalTok{)}
\NormalTok{\}}

\CommentTok{\# Calculate summary stats}
\NormalTok{mean\_adherent }\OtherTok{\textless{}{-}} \FunctionTok{mean}\NormalTok{(sample\_means\_adherent)}
\NormalTok{std\_adherent }\OtherTok{\textless{}{-}} \FunctionTok{sd}\NormalTok{(sample\_means\_adherent)}

\NormalTok{mean\_non\_adherent }\OtherTok{\textless{}{-}} \FunctionTok{mean}\NormalTok{(sample\_means\_non\_adherent)}
\NormalTok{std\_non\_adherent }\OtherTok{\textless{}{-}} \FunctionTok{sd}\NormalTok{(sample\_means\_non\_adherent)}

\FunctionTok{cat}\NormalTok{(}\StringTok{"Adherent Mean:"}\NormalTok{, mean\_adherent, }\StringTok{"}\SpecialCharTok{\textbackslash{}n}\StringTok{"}\NormalTok{)}
\end{Highlighting}
\end{Shaded}

\begin{verbatim}
## Adherent Mean: 34.51462
\end{verbatim}

\begin{Shaded}
\begin{Highlighting}[]
\FunctionTok{cat}\NormalTok{(}\StringTok{"Non{-}Adherent Mean:"}\NormalTok{, mean\_non\_adherent, }\StringTok{"}\SpecialCharTok{\textbackslash{}n}\StringTok{"}\NormalTok{)}
\end{Highlighting}
\end{Shaded}

\begin{verbatim}
## Non-Adherent Mean: 35.55352
\end{verbatim}

\begin{Shaded}
\begin{Highlighting}[]
\FunctionTok{cat}\NormalTok{(}\StringTok{"Adherent SD:"}\NormalTok{, std\_adherent, }\StringTok{"}\SpecialCharTok{\textbackslash{}n}\StringTok{"}\NormalTok{)}
\end{Highlighting}
\end{Shaded}

\begin{verbatim}
## Adherent SD: 3.876826
\end{verbatim}

\begin{Shaded}
\begin{Highlighting}[]
\FunctionTok{cat}\NormalTok{(}\StringTok{"Non{-}Adherent SD:"}\NormalTok{, std\_non\_adherent, }\StringTok{"}\SpecialCharTok{\textbackslash{}n}\StringTok{"}\NormalTok{)}
\end{Highlighting}
\end{Shaded}

\begin{verbatim}
## Non-Adherent SD: 1.954662
\end{verbatim}

\begin{Shaded}
\begin{Highlighting}[]
\CommentTok{\# Combine data for plotting}
\NormalTok{data\_for\_plot }\OtherTok{\textless{}{-}} \FunctionTok{data.frame}\NormalTok{(}
  \AttributeTok{mean\_values =} \FunctionTok{c}\NormalTok{(sample\_means\_adherent, sample\_means\_non\_adherent),}
  \AttributeTok{group =} \FunctionTok{rep}\NormalTok{(}\FunctionTok{c}\NormalTok{(}\StringTok{"Adherent"}\NormalTok{, }\StringTok{"Non{-}Adherent"}\NormalTok{), }\AttributeTok{each =}\NormalTok{ SAMPLES)}
\NormalTok{)}

\CommentTok{\# Plot the overlapping histograms}
\FunctionTok{ggplot}\NormalTok{(data\_for\_plot, }\FunctionTok{aes}\NormalTok{(}\AttributeTok{x =}\NormalTok{ mean\_values, }\AttributeTok{fill =}\NormalTok{ group)) }\SpecialCharTok{+}
  \FunctionTok{geom\_histogram}\NormalTok{(}\FunctionTok{aes}\NormalTok{(}\AttributeTok{y =} \FunctionTok{after\_stat}\NormalTok{(density)), }\AttributeTok{bins =} \DecValTok{30}\NormalTok{, }\AttributeTok{alpha =} \FloatTok{0.6}\NormalTok{,}
                 \AttributeTok{position =} \StringTok{"identity"}\NormalTok{, }\AttributeTok{color =} \StringTok{"white"}\NormalTok{) }\SpecialCharTok{+}
  \FunctionTok{labs}\NormalTok{(}\AttributeTok{title =} \StringTok{"Comparison of Sampling Distributions of Mean AHI (Adherent vs Non{-}Adherent)"}\NormalTok{,}
       \AttributeTok{x =} \StringTok{"Sample Mean AHI"}\NormalTok{, }\AttributeTok{y =} \StringTok{"Density"}\NormalTok{) }\SpecialCharTok{+}
  \FunctionTok{scale\_fill\_manual}\NormalTok{(}\AttributeTok{values =} \FunctionTok{c}\NormalTok{(}\StringTok{"lightgreen"}\NormalTok{, }\StringTok{"lightcoral"}\NormalTok{)) }\SpecialCharTok{+}
  \FunctionTok{theme\_minimal}\NormalTok{() }\SpecialCharTok{+}
  \FunctionTok{theme}\NormalTok{(}\AttributeTok{legend.title =} \FunctionTok{element\_blank}\NormalTok{())}
\end{Highlighting}
\end{Shaded}

\pandocbounded{\includegraphics[keepaspectratio]{Activity-2_trial1_files/figure-latex/unnamed-chunk-14-1.pdf}}

Interpretation: The average Apnea--Hypopnea Index (AHI) levels for both
groups are quite similar, ranging from 34 to 36. However, their sampling
distributions differ in spread. The non-adherent group shows a tighter,
more consistent distribution because it is based on a larger sample size
(n = 100), which reduces random variation.

In contrast, the adherent group's distribution is wider, as smaller
samples (n = 30) naturally produce more fluctuation in their averages.

In simple terms: Both groups have roughly the same average severity of
sleep apnea, but the results for non-adherent users are more stable and
predictable, while those for adherent users vary more from sample to
sample.

```

\end{document}
